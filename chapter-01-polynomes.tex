
\chapter{Polynômes}
\label{cha:polynomes}

Soit $R$ un anneau. On se souvient que ça veut dire que  $R$ est un ensemble,  muni avec des opérations binaires  $+ : R × R → R$  et $⋅: R × R → R$ tel que: 
\begin{enumerate}[(R1)]
\item $a+ b  = b+a$ pour tous     $a,b ∈ R$. \label{R1}
\item $a + (b+c) = (a + b) +c$, pour tous $a,b,c ∈ R$. \label{R2}
\item Il existe un élément $0 ∈R$ tel que $0+a =a$ pour chaque $a ∈R$. \label{R3}
\item Pour chaque $a ∈R$ il existe un élément $-a ∈R$ tel que $a + (-a) = 0$. \label{R4}
\item $a(bc) = (ab) c$ , pour tous $a,b,c ∈ R$. \label{R5}
\item Il existe un élément $1 ∈R$ tel que $a ⋅ 1 = 1 ⋅a$ pour chaque $a ∈R$. \label{R6}
\item $a (b+c) = ab + ac$ et $(b+c) a =ba +ca$  pour tous $a,b,c ∈R$.\label{R7} 
\end{enumerate}
Si on a en plus
\begin{enumerate}[(R1)]
  \setcounter{enumi}{7}
\item  $a b = ba$ pour tous $a,b ∈R$ \label{R8}. 
\end{enumerate}
alors l'anneau $R$ est appelé un \emph{anneau commutatif}.

\begin{exercise}
  Soit $R$ un anneau, alors l'élément $1$ est unique. 
\end{exercise}

\begin{example}
  \label{exe:32}
  \begin{enumerate}[i)]
  \item Les nombres entiers $ℤ$ avec l'addition et multiplication standard forment un anneau commutatif.
  \item $5 ⋅ℤ = \{ 5 z :z ∈ ℤ\}$ avec l'addition et multiplication standard n'est pas un anneau. La propriété \ref{R6}) n'est pas satisfaite.
  \item L'ensemble des matrices  $ℤ^{2 ×2}$ avec l'addition et multiplication des matrices est un anneau non commutatif. 
    \begin{eqnarray*}
      \begin{pmatrix}
        1 & 0 \\
        1 & 1 
      \end{pmatrix}
       \begin{pmatrix}
        1 & 1 \\
         0& 1 
       \end{pmatrix}   & = &
                            \begin{pmatrix}
                               1 & 1 \\
                               1 & 2 \\
                             \end{pmatrix}  \\      
      \begin{pmatrix}
        1 & 1 \\
        0 & 1 
      \end{pmatrix}
       \begin{pmatrix}
        1 & 0 \\
         1 & 1 
       \end{pmatrix}   & = &
                            \begin{pmatrix}
                               2 & 1 \\
                               1 & 1 \\
                             \end{pmatrix}       
    \end{eqnarray*}
  \end{enumerate}
\end{example}

Le théorème suivant est démontré dans le cours \emph{anneaux et corps}.   Il nous donne une manière formelle d'introduire le concept d'une indéterminée. 
\begin{theorem}
  \label{thr:50}
  Soit $R$ un anneau, alors il existe un anneau $S ⊇R$ et un élément $x ∈ S \setminus R$ tel que
  \begin{enumerate}[(i)]
  \item $a x = x a $ pour chaque $a ∈ R$.
  \item Si 
    $  a_0+ a_1x + \cdots + a_n x^n =0$  et $a_i ∈R$ pour tous $i$,  
    alors  $a_i = 0$ pour tous $i$.
  \end{enumerate}
\end{theorem}
Cet élément $x ∈ S \setminus R$ est appelé  \emph{indéterminée} ou  \emph{variable}.  

\begin{definition}
  \label{def:51}
  Un polynôme sur $R$ est une expression de la forme
$p(x) = a_0 + a_1 x + \cdots + a_n x^n$, $a_i ∈R$, $i=1,\dots,n$. L'élément $a_i$ est le $i$-ème \emph{coefficient} de $p(x)$. L'ensemble des polynômes sur $R$ est dénoté par $R[x]$. 
\end{definition}


\begin{example}
  \begin{enumerate}[i)]
  \item  $p(x) = 3 + x^2 + 5x^4 ∈ ℤ[x]$.
  \item $p(x) = \smat{1 &0 \\ 1 & 1} +   \smat{3 &3 \\ 2 & 1} x^3 ∈ ℤ^{2×2}[x]$. 
  \end{enumerate}
\end{example}

\begin{theorem}
  \label{thr:51}
  $R[x]$ est un anneau. Si $R$ est commutatif, alors $R[x]$ est commutatif. Si $R$ est anneau sans diviseur de zéro    alors $R[x]$ anneau sans diviseur de zéro. 
\end{theorem}
\begin{proof}
Pour la démonstration, on note que 
les conditions (R\ref{R1}) - (R\ref{R6}) sont satisfaites, parce que  $S$ est un anneau. Pour deux polynômes $f(x) = a_0 + a_1 x + \cdots + a_n x^n$ et $g(x) = b_0 + b_1x + \cdots + b_m x^m$, leur somme s'écrit comme
  \begin{equation}
    \label{eq:47}
    f(x) + g(x) = ∑_{i=1}^{\max\{m,n\}} (a_i +  b_i) x^i  ∈R[x]. 
  \end{equation}
  Alors   $R[x]$ est stable pour $+$. L'élément neutre de $(S,+)$, $0$ est un élément de $R[x]$.  
  L'inverse de $f(x) = a_0 + a_1 x + \cdots a_n x^n$ est $-f(x) = -a_0 - a_1 x  \cdots - a_n x^n$. Alors $(R[x],+)$ est un sous-groupe de $(S,+)$. 
Le    produit de $f$ et $g$ s'écrit comme
  \begin{equation}
    \label{eq:8}
    f(x) ⋅g(x) = ∑_{i=1}^{m+n} ( ∑_{j+k = i} a_j b_k) x^i ∈ R[x]. 
  \end{equation}
  C'est à dire que $R[x]$ est stable pour l'opération $⋅$ de $S$. Alors, $R[x]$ est un sous-anneau de $S$. 
  Alors $R[x]$ est un anneau. La formule~\eqref{eq:8} implique que
  $R[x]$ est commutatif, si $R$ est commutatif.

  Finalement, si $f(x) = a_0 + a_1 x + \cdots ≠ 0$ et
  $g(x) = b_0 + b_1 x + \cdots ≠ 0 $ sont deux polynômes et si $R$ est un 
  anneau sans diviseur de zéro, il faut montrer $f(x) ⋅ g(x) ≠
  0$. Soit $n = \max\{ i :a_i ≠0 \}$ et $m = \max\{ i :b_i ≠0 \}$. Le coefficient de $x^{m+n}$ du polynôme $f⋅g$ est $a_n ⋅b_n$. Ce coefficient n'es pas zéro des que $R$  est un 
  anneau sans diviseur de zéro. 
\end{proof}

\begin{example}
  \label{exe:33}
  \begin{eqnarray*}
    f(x) & = &  3 \, x^{3} + x + 2 \\
    g(x) & = & 3 \, x^{3} + 2 \, x^{2} + 1 \\
    f(x) ⋅g(x) & = & 9 \, x^{6} + 6 \, x^{5} + 3 \, x^{4} + 11 \, x^{3} + 4 \,  x^{2} + x + 2 \\ 
  \end{eqnarray*}
\end{example}

\emph{variable}.
D'un point de vue formel, $x ∈ R^* ⧹R$, où $R^*⊃R$ est un anneau, tel que
\begin{displaymath}
  a_0+ a_1x + \cdots + a_n x^n =0, \, a_i ∈K, \text{ implique } a_i=0 \text{ pour tous }i.  
\end{displaymath}
L'existence de $R$ et $x$ est traité en détail dans le cours \emph{anneaux et corps}. 
 Le polynôme $p(x)$ est écrit comme
\begin{displaymath}
  p(x) = a_0 + a_1x + a_2x^2 + \cdots
\end{displaymath}
où tous les  coefficients, sauf un nombre fini parmi eux, sont zéro.
Des expressions \eqref{eq:33} sont des polynômes sur $K$  et $K[x]$ est l'ensemble des polynômes sur $K$. Deux polynômes
\begin{equation}
  \label{eq:34}
  p(x) = a_0 + a_1x + a_2x^2 + \cdots \,\,\text{ et }  \,\, q(x) = b_0 + b_1x + b_2x^2 + \cdots
  \end{equation}
  sont \emph{égaux} si $a_i  =b_i$ pour tous $i$. Dans ce cas, on écrit $p(x) = q(x)$.
La \emph{somme} de $p(x)$ et $q(x)$~\eqref{eq:34} est le polynôme
\begin{displaymath}
  p(x) + q(x)  = a_0+b_0 + (a_1+b_1)x + (a_2+b_2)x^2 + \cdots
\end{displaymath}
et leurs \emph{produit} est
\begin{equation}
  \label{ceq:21}
  p(x) ⋅q(x) = a_0 b_0 + (a_0b_1 +a_1b_0) x + (a_0b_2+ a_1b_1 + a_2b_0)x^2 + \cdots .
\end{equation}
Le $i$-ème coefficient de $p(x)⋅q(x)$ est alors  $∑_{j+k=i}a_jb_k$.

\begin{theorem}
  \label{thr:43}
  L'ensemble des polynômes $K[x]$ sur un corps $K$ est un anneau intègre.
\end{theorem}

Le \emph{degré} de $p(x) = a_0 + a_1x + a_2x^2 + \cdots \neq 0$ est 
\begin{displaymath}
  \deg(p) = \max\{i \colon  a_i \neq 0\}
\end{displaymath}
et $\deg(0) = -\infty$. 
Si $p \neq 0$, le coefficient $a_{\deg(p)}$ est le \emph{coefficient dominant} de $p$. 
Un polynôme de degré zéro est une \emph{constante}. 

\begin{theorem}
  \label{thr:34}
  Pour $f,g \in K[x] $, $\deg(f \cdot g) = \deg(f) + \deg(g)$. 
\end{theorem}
\begin{proof}
  La formule~\eqref{ceq:21} révèle que $\deg(f\cdot g) \leq \deg(f) + \deg(g)$. 
  Soient $f(x) = a_0 + \cdots + a_n x^n$ et $g(x) = b_0+ \cdots b_m x^m$ tels que $a_n, b_m  \neq 0$. Le coefficient de $x^{n+m}$  est $a_n \cdot  b_m \neq 0$.
\end{proof}

Un polynôme $p(x) = a_0 + a_1 x + \cdots + a_n x^n ∈ K[x]$ induit une application $f_p:  K ⟶ K$, $f_p(k) = a_0+ a_1 k+ \cdots + a_n k^n$. Nous écrivons aussi $p(k)$ pour $f_p(k)$.  

\begin{theorem}
  \label{thr:42}
  Soit $K$ un corps infini. Deux polynômes $p(x),q(x) ∈ K[x]$ sont égaux, si et seulement si les applications $f_p$ et $f_q$ sont les mêmes.
\end{theorem}

La démonstration du théorème~\ref{thr:42} est le premier exercice maison.  

\begin{definition}
  \label{def:31}
  Soit $p(x) \in \K[x] \setminus\{0\}$. Un $\alpha \in K$ tel que $f_p(\alpha) = 0$ est une  \emph{ racine} de $f(x)$.  
\end{definition}


\begin{theorem}[Théorème fondamentale d'algebre]
  \label{thr:44}
  Tout polynôme $p(x) ∈ℂ[x] ⧹\{0\}$ de degré aux moins $1$  admet au moins une racine complexe.
\end{theorem}


L'expression \eqref{eq:32} est un polynôme avec indéterminée $λ$ et  comme polynôme formel, est défini par la formule de Leibniz
\begin{displaymath}
p_A(λ) = \det(A - λ I_n) = ∑_{π ∈S_n} \sign(π) ∏_{i=1}^n (A - λ I_n)_{iπ(i)}.
\end{displaymath}
Il est la somme des polynômes $ \sign(π) ∏_{i=1}^n (A - λ I_n)_{iπ(i)}$. Ça démontre que le degré de $p_A(λ)$ est au plus $n$. Mais le degré de
\begin{displaymath}
  \sign(\Id) ∏_{i=1}^n (A - λ I_n)_{i\Id(i)}
\end{displaymath}
est $n$ exacte. Ça démontre aussi que $a_n = (-1)^n$.

\begin{lemma}
  \label{lem:22}
Soit $p_A(λ) = a_0 + a_1 λ + \cdots + a_n λ^n$ le polynôme caractéristique de la matrice $A ∈ K^{n ×n}$. Alors, $a_0 = \det(A)$ et $a_n = (-1)^n$.
\end{lemma}


\begin{corollary}
  \label{co:10}
  Soit $V ≠ \{0\}$ un espace vectoriel de dimension fini sur $K = ℂ$, et $f: V → V$ un endomorphisme. Alors $f$ possède une valeur propre. 
\end{corollary}
\begin{proof}
  Soit $f(λ) ∈ ℂ[λ]$ le polynôme caractéristique de $f$ et $n$ la dimension de $V$. Le degré de $f$ est égal à $n≥1$, alors $p(x)$ possède une racine $λ^* ∈ℂ$. Cette racine $λ^*$  est une valeur propre de $f$. 
\end{proof}


La division avec reste est l'opération suivante. 

\begin{theorem}
  \label{thr:33}
  Soient $f,g \in K[x]$ et $\deg(g) >0$. Il existe $q,r \in K[x]$ unique  tels que 
  \begin{displaymath}
    f(x) = q(x) g(x) + r(x) 
  \end{displaymath}
  et $\deg(r) < \deg(g)$. 
\end{theorem}


\begin{proof}
  La preuve se fait par induction sur $\deg(f)$. Si $\deg(f) < \deg(g)$, alors on pose $q = 0$ et $r = f$.

Soit alors $\deg(f) = n \geq \deg(g)=m$ et 
\begin{displaymath}
  f(x) = a_0+ \cdots +a_n x^n \text{ et } g(x) = b_0 + \cdots + b_m x^m 
\end{displaymath}
où $a_n$ et $b_m$ sont les coefficients dominants de $f$ et $g$ respectivement. 
Clairement 
\begin{displaymath}
  \deg\left( f(x) - \frac{a_n}{ b_m } x^{n-m} g(x) \right) < \deg(f(x))
\end{displaymath}
et par induction 
\begin{displaymath}
  f(x) - \frac{a_n}{ b_m } x^{n-m} g(x)  = q(x) g(x) + r(x) 
\end{displaymath}
tel que $\deg(r(x)) < \deg(g(x))$. On  obtient alors
\begin{displaymath}
  f(x) = \left(q(x) + \frac{a_n}{ b_m } x^{n-m} \right) g(x) + r(x). 
\end{displaymath}

Supposons maintenant qu'il existent autres polynômes $q'(x)$ et $r'(x)$ tel que 
\begin{displaymath}
    f(x) = q'(x) g(x) + r'(x) 
  \end{displaymath}
  et $\deg(r') < \deg(g)$. 
Alors 
\begin{displaymath}
0 \neq   r(x) - r'(x) = (q(x) - q'(x)) ⋅ g(x). 
\end{displaymath}
On peut déduire 
\begin{displaymath}
\max\{\deg(r),\deg(r')\} \geq   \deg( r - r')  = \deg(q - q') + \deg(g) \geq \deg(g), 
\end{displaymath}
ce qui contredit le fait que $\deg(r) < \deg(g)$ et $\deg(r') < \deg(g)$. 
\end{proof}


% \begin{definition}
%   Pour $f(x)  = a_0 + \cdots + a_n x^n \in K[x]$ et $\alpha \in K$, l'évaluation $f(\alpha)$ est $ a_0 + a_1 \alpha + \cdots + a_n \alpha^n \in K$. 
% \end{definition}


\begin{example}
  \label{exe:24}
  Faire la division avec reste des polynômes  $x^5+2x^2+1$ par $2x^3+x+1$ de $ℤ_3[x]$ 
\end{example}

%\begin{proposition}
%  \label{prop:5}
%  Pour $\alpha \in K$, $\phi_\alpha: \, K[x] \longrightarrow K$, %$\phi_\alpha(f(x)) = f(\alpha)$ est un homomorphisme.  
%\end{proposition}


\begin{definition}
  \label{def:32}
  Un polynôme  $q(x) ∈K[x]$ \emph{divise} un  polynôme $f(x)∈ K[x]$ s'il existe un polynôme $g(x)$ tel que $f(x) = g(x) \cdot q(x)$. On dit que $q(x)$ est un diviseur de $f(x)$ et on écrit $q(x) \mid f(x)$. 
\end{definition}

\begin{theorem}
  \label{thr:35}
  Soit $f(x)∈ K[x] \setminus \{0\}$ un polynôme  et $\alpha \in K$, alors $\alpha$ est une racine de $f$ si et seulement si $(x- \alpha)  \mid f(x)$. 
\end{theorem}

\begin{proof}
  Si $f(x) = q(x) \cdot (x - \alpha)$, alors $f(\alpha) = 0$. 

%Autrement, si $f$ est une constante, $f = 0$ et $(x - \alpha)$ divise $f$.
Dans l'autre sens, si $f$ est une constante, $f(\alpha) = 0$ implique que $f = 0$ et $(x - \alpha)$ divise $f$. 

Si $f$ n'est pas une constante, il existe $q(x)$ et $r(x)$ tels que
\begin{displaymath}
  f(x) = q(x) \cdot (x - \alpha) + r(x)
\end{displaymath}
%$\deg(r) = 0$. Alors $f(\alpha) = 0$ implique $r=0$. 
avec $\deg(r) \leq 0$. Alors $f(\alpha) = 0$ implique $r=0$. 
\end{proof}


\begin{definition}
  \label{def:41}
  La \emph{multiplicité} d'une racine $α$ de $p(x) ∈ K[x] ⧹\{0\}$ est le plus grand $i≥1$ tel que $ (x-α)^i \mid p(x)$. Si $p(x)$ est le polynôme caractéristique d'un endomorphisme d'un espace vectoriel, on appelle la multiplicité de $α$ la \emph{multiplicité algébrique}. 
\end{definition} 



%%% Local Variables:
%%% mode: latex
%%% TeX-master: "notes"
%%% End:

\chapter{Le théorème spectral et la décomposition en valeurs singulières}

\label{cha:appl-auto-adjo}

Dans ce chapitre, nous allons étudier les espaces euclidiens et hermitiens d'une manière plus profonde. Lorsque l'on parle de $\C$ ou de $\R$, nous allons utiliser la lettre $\K$ pour dénoter $\C$ ou $\R$. On va rappeler quelques notions importantes du cours du premier semestre. Un \emph{endomorphisme}  est une application linéaire $f \colon V \longrightarrow V$. Si $V$ est un espace vectoriel de dimension finie et si $B = \{v_1,\dots,v_n\}$ est une base de $V$, on a 
\begin{displaymath}
  f(x) = \phi_B^{-1} (A_B \phi_B(x)),
\end{displaymath}
où $\phi_B$ est l'ismomorphisme $\phi_B \colon V \longrightarrow K^n$, $\phi_B(x) = [x]_B$ sont les coordonnées de $x$ par rapport à la base $B$. On a le diagramme suivant 
\begin{displaymath}
  {
  \begin{CD}
    V     @>f>>  V\\
    @VV \phi_B V        @VV \phi_B V\\ 
    K^n     @>A_B \cdot x>>  K^n
  \end{CD}} 
\end{displaymath} 
Les colonnes de la matrice $A_B$ sont les coordonnées de $f(v_1),\dots,f(v_n)$ dans la base $B$. 
Une matrice $A \in K^{n \times n}$ est \emph{diagonalisable} s'il existe une matrice inversible $P \in K^{n \times n}$ telle que $P^{-1}\cdot A \cdot P$ est une matrice diagonale. 

\section{Les endomorphismes auto-adjoints} 
\label{sec:les-endom-auto}
\begin{framed}\noindent 
  Dans ce paragraphe~\ref{sec:les-endom-auto},  $V$
  est toujours un espace euclidien ou un espace hermitien de dimension finie. 
\end{framed}

\begin{definition}
\label{def:19}
Un endomorphisme $F$ est \emph{auto-adjoint} si 
\begin{displaymath}
  \pscal{F(v),w} = \pscal{v,F(w)} \text{ pour tous } v,w \in V. 
\end{displaymath}
\end{definition}

\begin{theorem}
  \label{thr:14}
  Soient $B = \{v_1,\dots,v_n\}$ une base orthonormale de $V$ et $F$ un endomorphisme. Alors $F$ est auto-adjoint si et seulement si sa matrice $A_B$ dans la base $B$ est symétrique ($\K = \R$) ou hermitienne ($\K = \C$). 
\end{theorem}

\begin{proof}
  On traite seulement le cas $\K=\C$. Le cas $\K = \R$ est démontré d'une manière analogue. Nous avons, où $\cdot$ dénote le produit hermitien standard, 
  \begin{displaymath}
    \pscal{F(v), w} = (A_B [v]_B ) \cdot [w]_B = [v]_B^T A_B^T \overline{[w]_B},
  \end{displaymath}
et 
\begin{displaymath}
  \pscal{v, F(w)}  = [v]_B^T \overline{A_B} \overline{[w]_B}. 
\end{displaymath}
Alors si $\overline{A_B} = A_B^T$, il est clair que $\pscal{F(v), w} = \pscal{v, F(w)}$ et donc $F$ est auto-adjoint. \\
Et si $F$ est auto-adjoint, en choisissant $v = v_i$, $w = v_j$, on obtient 		\begin{displaymath}
	\pscal{F(v_i), v_j} = e_i^TA_B^Te_j = (A_B^T)_{i, j} = \pscal{v_i, F(v_j)} = e_i^T \overline{A_B} e_j = (\overline{A_B})_{i, j},
	\end{displaymath}
donc $A_B^T = \overline{A_B}$
\end{proof}


\begin{lemma}
  \label{lem:8}
  Soit $A \in \C^{n \times n}$ une matrice hermitienne. Les valeurs propres de $A$ sont réelles. 
\end{lemma}
\begin{proof}
  Soient $\lambda \in \C$ une valeur propre et $v \neq 0$ son vecteur propre. Alors 
  \begin{displaymath}
    \lambda \, v^T \overline{v}  = v^T {A}^T \overline{v} = 
    v^T \overline{A \, v} = \overline{\lambda}  \, v^T \overline{v}. 
  \end{displaymath}
\end{proof}


\begin{corollary}
\label{co:3}
Soit $F$
un endomorphisme auto-adjoint, alors toutes ses valeurs propres sont
réelles.
\end{corollary}
\begin{proof}
  Soit $B = \{v_1,\dots,v_n\}$
  une base orthonormale. Les valeurs propres de $F$
  sont les valeurs propres de la matrice hermitienne $A_B$.
\end{proof}

\begin{corollary}
  \label{co:7}
  Une matrice symétrique $A \in \R^{n \times n}$ (hermitienne $A \in \C^{n \times n}$) a une valeur propre réelle. 
\end{corollary}

\begin{proof}
  Le polynôme caractéristique $p(x) = \det(A - x \cdot I_n)$ a une racine complexe selon le théorème fondamental de l'algèbre. Les valeurs propres de $A$ sont les racines de $p(x)$.  Mais toutes ces racines sont réelles selon le         corollaire~\ref{co:3}.  
\end{proof}

\begin{remark}
La même preuve, par le théorème fondamental de l'algèbre, montre en fait que le polynôme caractéristique de $A$ est scindé sur $\Bbb R$, i.e. $A$ possède $n$ valeurs propres réelles (en comptant les multiplicités algébriques).
\end{remark}

\begin{lemma}
  \label{lem:9}
  Soient $F$ un endomorphisme auto-adjoint et $u,v \neq 0$ deux vecteurs propres dont leurs valeurs propres  sont différentes, alors $\pscal{u,v}=0$. 
\end{lemma}

\begin{proof}
  Soient $\lambda \neq \gamma $ les valeurs propres correspondant aux vecteurs propres $u,v \neq 0$ respectivement. Puisque $\lambda,\gamma \in \R$ on a
  \begin{displaymath}
    \lambda \pscal{u,v} = \pscal{F(u),v} = \pscal{u,F(v)} = \gamma \pscal{u,v}
  \end{displaymath}
et alors $\pscal{u,v}=[u]_B \cdot [v]_B=0$, où $\cdot$ dénote le produit scalaire/hermitien standard et $B$ est une base orthonormale de $V$.  
\end{proof}



\begin{definition}
  \label{def:20}
  Une matrice inversible  $U \in \R^{n \times n}$ est \emph{orthogonale} si $ U^{-1}= U^T  $.  Une matrice inversible $U \in \C^{n \times n}$ est \emph{unitaire} si $U^{-1} = \overline{U}^T$. 
\end{definition}
\noindent 
Si $U$ est orthogonale (unitaire), les  colonnes de $U$ sont une base orthonormale de $\R^n$ ($\C^n$), où l'orthonormalité est entendue au sens du produit scalaire (hermitien) standard. 
\begin{notation}
Nous allons écrire $A^*$ pour dénoter $\overline{A}^T$ pour une matrice $A$.   
\end{notation}




\begin{theorem}[Théorème spectral]
\label{thr:16}
  Soit $A \in \K^{n\times n}$ une matrice symétrique (hermitienne), alors $A$ est diagonalisable avec une matrice orthogonale (unitaire) $P \in \K^{n \times n}$ telle que 
  \begin{equation}
    \label{diagonal}
    P^* \cdot A \cdot P =
    \begin{pmatrix}
      \lambda_1 \\
      & \ddots \\
      && \lambda_n
    \end{pmatrix}
  \end{equation}
  où $\lambda_1,\dots,\lambda_n \in \R$ sont les valeurs propres de $A$. 
\end{theorem}


\begin{proof}
  Soit $A \in \R^{n\times n}$
  une matrice symétrique. Le cas où $A \in \C^{n \times n}$
  est hermitienne est laissé en exercice.

  Le théorème est démontré par induction. Si $n=1$, l'assertion est triviale. \\
  Supposons le théorème vrai jusqu'à $n-1 \in \N, n \ge 2$. Montrons le pour $n$. \\
  Soit $\lambda_1 \in \R$
  une valeur propre de $A$
  et $v \neq 0 \in \R^n$
  un vecteur propre correspondant à $\lambda_1$.
  Avec la méthode de Gram-Schmidt, on peut trouver une base
  orthonormale $\{v,u_2,\dots,u_n\}$ dans $\R^n$. Soit $U \in \R^{n \times n-1}$ la matrice dont les colonnes sont $u_2,\dots,u_n$.  On considère la matrice
  \begin{displaymath}
    U^T \cdot A \cdot U \in \R^{n-1 \times n-1}.
  \end{displaymath}
  Cette matrice est symétrique. En effet :
  \begin{displaymath}
    (U^T \cdot A \cdot U)^T = U^T \cdot A^T \cdot (U^T)^T = U^T \cdot A \cdot U,
  \end{displaymath}
  car $A$ est symétrique. Par hypothèse d'induction, cette matrice peut être diagonalisée  avec une matrice orthogonale  $K \in \R^{ n-1 \times n-1}$, alors 
  \begin{displaymath}
    K^T \cdot  U^T \cdot A \cdot U \cdot K =
    \begin{pmatrix}
      \lambda_2 \\
      & \ddots \\
      && \lambda_n
    \end{pmatrix}. 
  \end{displaymath}
Maintenant, soit $P \in \R^{n \times n}$ la matrice 
\begin{displaymath}
  P = \left( v, U\,K\right) \in \R^{n \times n}
\end{displaymath} 
La matrice $P$ est orthogonale puisque 
\begin{displaymath}
  P^T \, P =
  \begin{pmatrix}
    v^T \, v & v^T U \, K \\
    (U\,K)^T \, v &  (U\,K)^T (U\,K)
  \end{pmatrix} = 
  \begin{pmatrix}
    1 \\
    & \ddots \\
    && 1 
  \end{pmatrix}. 
\end{displaymath}
car $v^T \cdot v = 1$, $U^T \cdot U = I_{n-1}$ et $v^T \cdot U = 0$. Et 
\begin{displaymath}
  P^T \, A \, P =
  \begin{pmatrix}
    v^T A \\
    K^T U^T A 
  \end{pmatrix}
  \begin{pmatrix}
    v & U\,K
  \end{pmatrix} =
  \begin{pmatrix}
    \lambda_1 \\
    & \ddots \\
    && \lambda_n
  \end{pmatrix}  
\end{displaymath}
\end{proof}


\begin{corollary}
  \label{co:8}
  Soit $V$ un espace euclidien (hermitien) de dimension finie et $F$ un endomorphisme auto-adjoint. Alors $V$ possède une base $\{v_1,\dots,v_n\}$ orthonormale de vecteurs propre de $F$. 
\end{corollary}

\begin{proof}
  Soit $B = \{u_1,\dots,u_n\}$ une base de $V$ tel que $\pscal{x,y} = [x]_B \cdot [y]_B$ où $\cdot $ dénote le produit hermitien standard (voir chapitre~\ref{sec:espaces-hermitiens}, exercice~\ref{item:4}) et soit $A_B^{(F)}$  la matrice symétrique (hermitienne) telle que $F(x)  = \phi_B^{-1}(A_B^{(F)} [x]_B)$. Selon théorème~\ref{thr:16}, $A_B^{(F)}$ est diagonalisable avec une matrice orthogonale (unitaire) $P$ 
  \begin{displaymath}
    P^* \cdot 
    A_B^{(F)} \cdot  P  = \begin{pmatrix}
      \lambda_1 \\
      & \ddots \\
      & & \lambda_n
    \end{pmatrix}
  \end{displaymath}
 Soient $p_1,\dots,p_n$  les colonnes de $P$. La base orthonormale de vecteurs propres de $F$ est $\{v_1,\dots,v_n\}$ où $v_i = \phi_B^{-1}(p_i)$. 
\end{proof}



Comment peut-on calculer la diagonalisation \eqref{diagonal}? Voici un procédé pour diagonaliser une matrice symétrique (hermitienne) $A \in \K^{n \times n}$. 

\begin{enumerate}[i)] 
\item Trouver les racines $\lambda_1,\dots,\lambda_k \in \R$ du polynôme caractéristique 
  \begin{displaymath}
    p(x) = \det(A - x\cdot I). 
  \end{displaymath}
\item Pour tout $j \in \{1,\dots,k\}$:
  \begin{enumerate}[a)] 
  \item Trouver une base $b^{(j)}_1,\dots,b^{(j)}_{d_j}$ du noyau de la matrice $A - \lambda_j \, I$, par exemple avec l'algorithme de Gauss. 
  \item Trouver une base orthonormale $p^{(j)}_1,\dots,p^{(j)}_{d_j}$ du $\spa\{b^{(j)}_1,\dots,b^{(j)}_{d_j}\}$, par exemple avec le procédé de Gram-Schmidt. 
  \end{enumerate}
  \item $P = \left(p^{(1)}_1,\dots,p^{(1)}_{d_1},\dots,p^{(k)}_1,\dots,p^{(k)}_{d_k}\right)$ 
\end{enumerate}




\begin{example}
  \label{exe:17}
  Soit $V$ un espace euclidien de dimension $3$ et soit $F$ un endomorphisme auto-adjoint de $V$. Soit $B=\{v_1,v_2,v_3\}$ une base orthonormale telle que 
  \begin{displaymath}
    A_B =
    \begin{pmatrix}
      3 & -2 & 4\\-2 & 6 & 2\\4 & 2 & 3
    \end{pmatrix}
  \end{displaymath}
Trouver une base orthonormale qui se compose des vecteurs propres. 


\bigskip 

Le polynôme caractéristique de $A_B$ est 
\begin{displaymath}
  p(x) = \det(A_B - x\, I) =  -x^3 + 12\, x^2 -21 x - 98 = -(x-7)^2(x+2)
\end{displaymath}
On trouve les bases des espaces propres
\begin{displaymath}
  \lambda_1 = 7: \, \,
b_1^{(1)} = \begin{pmatrix}
    1 \\ 0 \\ 1 
  \end{pmatrix}, \, 
b_2^{(1)} =   \begin{pmatrix}
    -1/2\\1\\0
  \end{pmatrix} \, \quad \text{ et } 
  \lambda_2 = -2: \, \, 
  b_1^{(2)} = 
  \begin{pmatrix}
    -1 \\ -1/2 \\1
  \end{pmatrix}  
\end{displaymath}

Les vecteurs $b_1^{(1)}$
et $b_2^{(1)}$
ne sont pas orthogonaux.  Le procédé de Gram-Schmidt produit
${b_2^{(1)}}^* = (-1/4,1,1/4)^T$.
Les vecteurs $b_1^{(1)},{b_2^{(1)}}^*,b_1^{(2)}$
sont une base orthogonale de vecteurs propres. Maintenant il reste à
les normaliser et on obtient
\begin{displaymath}
p_1^{(1)} = 
\left[\begin{matrix}\frac{\sqrt{2}}{2}\\0\\\frac{\sqrt{2}}{2}\end{matrix}\right], \, 
{p_2^{(1)}} = \left[\begin{matrix}- \frac{\sqrt{2}}{6}\\\frac{2 \sqrt{2}}{3}\\\frac{\sqrt{2}}{6}\end{matrix}\right], \, 
p_1^{(2)} = \left[\begin{matrix}-\frac{2}{3}\\-\frac{1}{3}\\\frac{2}{3}\end{matrix}\right]. 
\end{displaymath}

Alors $\left\{\frac{\sqrt{2}}{2}v_1 + \frac{\sqrt{2}}{2} v_3,- \frac{\sqrt{2}}{6}v_1 + \frac{2 \sqrt{2}}{3}v_2 + \frac{\sqrt{2}}{6} v_3 , -\frac{2}{3}v_1-\frac{1}{3}v_2+\frac{2}{3}v_3\right\}$ est une base orthonormale de vecteurs propre de $F$. 

\end{example}

\subsection*{Exercices}

\begin{enumerate}
\item Soit $z = x + iy \in \C^n$, où $x,y \in \R^n$. Montrer que $x$ et $y$ sont linéairement indépendants sur $\R$ si et seulement si $z$ et $\overline{z}$ sont linéairement indépendants sur  $\C$. 
\end{enumerate}


\section{Formes quadratiques réelles et matrices symétriques réelles}
\label{sec:form-quadr-reell}

Le lemme~\ref{lem:8} et le corollaire~\ref{co:7} démontrent qu'une matrice symétrique réelle possède une valeur propre réelle. Cette démonstration passe par les nombres complexes et utilise le théorème fondamental de l'algèbre. Pour le cas où $A = A^T \in \R^{n \times n}$, nous allons maintenant démontrer l'assertion du  corollaire~\ref{co:7}  d'une manière géométrique. L'ensemble 
\begin{displaymath}
  S^{n-1} = \{x \in \R^n \colon \|x\| = 1 \} 
\end{displaymath}
est appelé la \emph{$n$-sphère.} 

\begin{definition}
  \label{def:21}
  Une \emph{forme quadratique} est une fonction
  $f\colon \R^n \longrightarrow \R$,
  $f(x) = x^T A x$
  où $A \in \R^{n \times n}$ est une matrice symétrique.
\end{definition}
En fait, si $B \in \R^{n \times n}$
n'est pas symétrique, la matrice $A = 1/2 (B^T + B)$
est symétrique et $x^TBx = 1/2 (x^TB^T x + x^T B x) = x^T Ax$.
Alors la fonction $g(x) = x^TBx$ est aussi une forme quadratique.

Une {forme quadratique} est un polynôme de degré $2$
et une fonction continue. Puisque $S^{n-1}$
est compact et $f(x)$
est continue, $f(x)$
possède un maximum sur $S^{n-1}$.
Nous sommes prêts à démontrer le lemme d'une manière géométrique.

\begin{lemma}
  \label{lem:7}
  Soient $A \in \R^{n \times n}$ une matrice symétrique et $v \in S^{n-1}$ le maximum de la fonction $f(x) = x^TAx$ sur $S^{n-1}$. On a $Av = \lambda\, v$ pour  un $\lambda \in \R$. En particulier, $A$ possède une valeur propre réelle. 
\end{lemma}

\begin{proof}
  Supposons qu'il n'existe pas de $\lambda \in \R$  tel que $A\,v  =  \lambda v$. Alors,  en particulier, $A\, v \neq 0$ et on peut écrire 
  \begin{displaymath}
    A\,v = \alpha \, v + \beta \, w
  \end{displaymath}
où $w \in S^{n-1}$, $w \perp v$ et $\beta \neq 0$. Pour $x \in [-1,1]$ on a 
\begin{displaymath}
  \sqrt{(1 - x^2)} \,v + x\, w \in S^{n-1}.
\end{displaymath}On voit facilement que $\|\sqrt{(1 - x^2)} \,v + x\, w\|=1$ en utilisant le fait que $v\perp w$, et $v,w\in S^{n-1}$.
Nous considérons  la fonction $g\colon\; [-1,1] \rightarrow \R$ 
\begin{displaymath}
  g(x) = \left(\sqrt{(1 - x^2)} \,v + x\, w \right)^T A \left(\sqrt{(1 - x^2)} \,v + x\, w \right). 
\end{displaymath}
Notons que $g(0)=f(v)$, donc si $v$ maximise $f$ sur la $n$-sphère, en particulière $x=0$ doit maximiser $g(x)$ dans l'intervalle [-1,1]. Si on démontre que $g'(0) \neq 0$, nous avons déduit une contradiction et la démonstration est faite. 

Comme $w^TAv = v^TAw$, clairement 
\begin{displaymath}
  g(x) = (1-x^2) v^T A v + (2 \cdot \sqrt{(1 - x^2)} \cdot x) \, w^T A v + x^2 w^TAw.
\end{displaymath}
Ceci démontre que $g'(0) = 2 \cdot w^T A v = 2 \cdot \beta  \neq 0$. 

\end{proof}





\begin{definition}
  \label{def:f22}
  Une matrice symétrique $A \in \R^{n \times n}$ est 
  \begin{itemize}
  \item définie positive, si $x^TAx>0$ pour tout $x \in \R^n \setminus  \{0\}$ 
  \item définie négative, si $x^TAx<0$ pour tout $x \in \R^n \setminus  \{0\}$ 
  \item semi-définie positive, si $x^TAx\geq 0$ pour tout $x \in \R^n$ 
  \item semi-définie négative, si $x^TAx\leq 0$ pour tout $x \in \R^n$.  
  \end{itemize}
  La forme quadratique $x^TAx$ correspondante est appelée définie positive, définie négative, semi-définie positive ou semi-définie négative, en accord avec $A$. 
\end{definition}



\begin{theorem}
  \label{thr:15}
  Une matrice symétrique $A \in \R^{n \times n}$ est 
  \begin{enumerate}
  \item définie positive,
    si et seulement si toutes ses valeurs propres sont strictement positives. 
  \item définie négative, si et seulement si toutes ses valeurs propres sont strictement négatives. 
  \item semi-définie positive, si et seulement si toutes ses valeurs propres sont positives (ou zéro).  
  \item semi-définie négative, si et seulement si toutes ses valeurs propres sont negatives (ou zéro).  
  \end{enumerate}
\end{theorem}

\begin{proof}
  D'après le théorème~\ref{thr:16} il existe une 
  matrice orthogonale $U \in \R^{n\times n}$ telle que 
  \begin{equation}
    \label{eq:10}    
    A = U
    \begin{pmatrix}
      \lambda_1 \\
      & \ddots \\
      && \lambda_n
    \end{pmatrix} U^T. 
  \end{equation}
  où les $\lambda_i$ sont les valeurs propres de $A$. Les colonnes  $u_1,\dots,u_n$ de $U$ forment une base orthonormale de $\R^n$  de vecteurs propres de $A$. 
  Soit $x \in \R^n$, alors $x = \sum_i \alpha_i u_i$ et 
  \begin{displaymath}
    x^TAx = \sum_i (\alpha_i^2) \lambda_i.  
  \end{displaymath}
D'ici l'assertion suit directement. 
\end{proof}

Le \emph{$k$-mineur principal} d'une matrice $A$ est le déterminant de la matrice qui est construite en choisissant les premières $k$ lignes et colonnes de $A$; c'est-à-dire $\det(B_k)$ où $B_k \in \R^{k\times k}$ telle que $b_{ij} = a_{ij}$, $1 \leq i,j \leq k$. Soit $K  = \{l_1,\dots,l_k\}\subseteq \{1,\dots,n\}$ où $l_1<l_2<\dots<l_k$. 
La matrice $B_K \in \R^{k\times k}$  est définie par $b_{ij} = a_{l_il_j}$, pour $1 \leq i,j\leq k$. Un \emph{$k$-mineur symétrique} de $A$ est le déterminant d'une matrice $B_K$; c'est-à-dire $\det(B_K)$.  

\begin{theorem}
  \label{thr:17}
  Soit $A \in  \R^{n \times n}$ une matrice symétrique.   
  \begin{enumerate}[a)]
  \item $A$ \label{posdef:1}
    est définie positive si et seulement si tous ses mineurs
    principaux sont strictement positifs.
  \item $A$ \label{posdef:2}
    est semi-définie positive si et seulement si tous ses mineurs symétriques sont non
    négatifs.
  \end{enumerate}
\end{theorem}

\begin{proof}
%Nous   considérons la factorisation~\eqref{eq:10}. La 

On démontre~\ref{posdef:1}), tandis que \ref{posdef:2}) est une exercice. 
Si $A$ est définie positive, alors 
les matrices $B_k$, $ 1 \leq k \leq n$,  sont symétriques  et définies positive. Leurs valeurs propres sont toutes positives.  Selon le théorème~\ref{thr:16}, $\det(B_k)$ est le produit des valeurs propres de $B_k$, alors $\det(B_k)>0$. 


Supposons maintenant que $\det(B_k)>0$ pour tout $k \in \{1,\dots,n\}$. Nous appliquons notre algorithme~\ref{alg:1}. Avant la $i$-ème itération, la matrice $A$ est de la forme 
\begin{displaymath}
  P^T A P = \begin{pmatrix}
      c_1 \\
      & c_2 \\
      & & \ddots & &&\\
      & & & c_{i-1} \\
      & & & &  b_{i,i} & \dots & b_{i,n} \\
%      & & & &  b_{i+1,i} & \dots & b_{i+1,n} \\
      & & & &     \vdots       &  & \vdots \\
      & & & &  b_{n,i} & \dots & b_{n,n} \\      
    \end{pmatrix}. 
\end{displaymath} 
 On observe que $b_{ii} \neq 0$. Parce que, si $b_{ii} = 0$ pour la première fois, nous n'avons jamais échangé de colonnes ou de lignes avant et  $ 0 = c_1\cdots c_{i-1} \cdot b_{ii} = \det(B_i)$, et c'est une contradiction.  
 Alors il n'est jamais nécessaire d'échanger des lignes et colonnes et notre algorithme trouve une matrice $R$, triangulaire supérieure, dont les éléments diagonaux sont tous égaux à $1$, et telle que 
\begin{displaymath}
  R^T \cdot A \cdot R =  
\begin{pmatrix}
      c_1 \\
             & \ddots & \\
       & & c_{n}
    \end{pmatrix}
\end{displaymath}
On observe que $\det(B_k) = c_1\cdots c_k$, alors toutes les $c_i$ sont positives. Alors $A$ est définie positive. 
\end{proof}


\begin{theorem}
  \label{thr:18}
  Soit $A \in \R^{n\times n}$ une matrice symétrique et $f(x) = x^TAx$ la forme quadratique correspondante à $A$. On a
  \begin{equation}
    \label{eq:11}
    \max_{x \in S^{n-1}} f(x) = \lambda_1  \, \text{ et } \,  \min_{x \in S^{n-1}} f(x)  = \lambda_n
  \end{equation}
  où $\lambda_1$ et $\lambda_n$ sont les valeurs propres maximale et minimale de $A$ respectivement. 
\end{theorem}

\begin{proof}
  Nous utilisons la factorisation 
  \begin{displaymath}
    A = P
    \begin{pmatrix}
      \lambda_1 \\
      & \ddots \\
      && \lambda_n
    \end{pmatrix} P^T
  \end{displaymath}
où $P \in \R^{n\times n}$ est une matrice orthogonale dont les colonnes sont $p_1,\dots,p_n$. Si $x = \sum_i \alpha_i \,p_i$,   alors 
\begin{displaymath}
\|x\|^2 = \sum_i \alpha_i^2   \text{ et } x^T A x =  \sum_i (\lambda_i \alpha_i^2)
\end{displaymath}
et si $\|x\|^2 = 1$, 
\begin{displaymath}
\lambda_n =  \lambda_n  \sum_i  \alpha_i^2  \leq  \sum_i (\lambda_i \alpha_i^2) \leq \lambda_1  \sum_i  \alpha_i^2 = \lambda_1. 
\end{displaymath}
Ça démontre que $p_1$ et $p_n$ sont les solutions optimales des problèmes d'optimisation~\eqref{eq:11}. 

\end{proof}



\begin{definition}
  \label{def:23}
  Soit $A \in \R^{n \times n}$ une matrice symétrique. Pour $x \in \R^n\setminus \{0\}$ le  \emph{quotient Rayleigh–Ritz } est 
\begin{displaymath}
  R_A(x) = \frac{x^TAx}{x^Tx}. 
\end{displaymath}
\end{definition}
Pour $x \in \R^n \setminus \{0\}$ $ x / \|x\| \in S^{n-1}$ et $R_A(x) = (x / \|x\| )^T \cdot A  (x / \|x\|)$. 

\begin{theorem}[Théorème Min-Max]
\label{thr:19}
Soit $A \in \R^{n × n}$ une matrice symétrique avec les valeurs propres 
$\lambda_1 \geq \dots \geq \lambda_n$.   Si $U$ dénote un sous-espace de $\R^n$ alors 
\begin{equation}
  \label{eq:12} 
  \lambda_k = \max_{ \dim(U) = k } \, \min_{x \in U \setminus \{0\}}  R_A(x)  
\end{equation}
et
\begin{equation}
  \label{eq:13}
  \lambda_k = \min_{ \dim(U) = n-k+1 } \, \max_{x \in U \setminus \{0\}}  R_A(x)  
\end{equation}
\end{theorem}

\begin{proof}
Nous démontrons \eqref{eq:12}. La partie \eqref{eq:13} est un exercice. Soit
$\{u_1,\dots,u_n\}$ une base orthonormale de vecteurs propres associés à
$\lambda_1 ≥ \dots ≥ \lambda_n$ respectivement. On fixe un entier $k$, soit $U$
un sous-espace de dimension $k$. Posons $W := \spa\{u_k, \dots, u_n\}$.
Comme
\begin{equation*}
	\dim(W \cap U) = \dim(U) + \dim(W) - \dim(U + W) \geq k + (n - k + 1) - n = 1,
\end{equation*}
$W \cap U$ contient un élement $x$ non-nul. On a que $R_A(x) \leq \lambda_k$ et
donc pour tout sous-espace $U$ de dimension $k$, $\min_{x \in U \setminus \{0\}}
R_A(x) \leq \lambda_k$. Or pour $U = \spa\{u_1,\dots,u_k\}$, $\min_{x \in U \setminus \{0\}}  R_A(x) = \lambda_k.$  Ainsi on a démontré \eqref{eq:12}. 
\end{proof}



\subsection*{Exercices}

\begin{enumerate}
\item Une matrice réelle  symétrique, différente de la matrice zéro, telle que toute composante sur la diagonale est zéro ne peut pas être semi définie positive, ni semi définie négative. 
\item Une matrice réelle symétrique, différente de la matrice zéro, dont la diagonale est égale à zéro, possède un $2 ×2$ mineur symétrique négatif. \label{item:17}
\item Soit $L \in ℝ^{n × n}$ une matrice %triangulaire inférieure 
de la forme 
  \begin{displaymath}
L = \left(\begin{array}{c|c}
H & 0  \\
\hline
C  & I_{n-i+1} \\
\end{array}\right)
\end{displaymath}
où  $H ∈ ℝ^{i×i}$ et $i ≥0$. Soit $Q ∈ ℝ^{n ×n}$ la matrice de la permutation (transposition)  qui échange $μ, ν > i$. Montrer 
\begin{displaymath}
  Q \cdot L = \left(\begin{array}{c|c}
H & 0  \\
\hline
C'  & I_{n-i+1} \\
\end{array}\right) \cdot Q
\end{displaymath}
où $C'$ provient de $C$ en échangeant les lignes $μ-i$ et $ν-i$. 
\label{item:15}
\item \label{item:16}
En s'appuyant sur l'exercice~\ref{item:15}) montrer l'assertion suivante. Si l'algorithme~\ref{alg:1} a exécuté $k$-itérations et dans chacune de ces $k$ itérations, il existe un $j ≥i$ tel que  $b_{jj} \neq 0$ (avec la notation de la $i$-ème itération), alors il existe une matrice de permutation $Q$ telle que le résultat de ces premières $k$ itérations s'écrit 
\begin{displaymath}
  R^T Q^T A Q R, 
\end{displaymath}
où $R$ est une matrice triangulaire supérieure dont les éléments diagonaux sont $1$.  En autres mots, il existe une permutation si appliquée aux lignes et colonnes, l'algorithme~\ref{alg:1} n'échange pas de lignes et colonnes pendant ces premiers $k$ itérations. 
\item Soient $A \in \Bbb R^{n \times n}$ réelle symétrique et $A'$ la matrice obtenue de $A$ en échangeant les lignes $i$ et $j$ ($A' = Q^T A Q$ pour une matrice de permutation (transposition) $Q$). Montrer que pour tout $K \subseteq \{1, ..., n\}$, il existe $K' \subseteq \{1, ..., n\}$ tel que $\det(A_K) = \det(A'_{K'})$ (mineurs symétriques) et inversément. En d'autres termes, montrer que tous les mineurs symétriques de $A$ se retrouvent dans $A'$ et vice versa.
\item Montrer la partie b) du théorème~\ref{thr:17}. 

\emph{Indication pour montrer $⟸$:  Il existe une matrice de permutation $Q$ telle que l'algorithme~\ref{alg:1} n'échange pas de colonnes et lignes si confronté avec  $Q^T \cdot A \cdot Q$ comme input et toutes itérations sont telles que $b_{ii} \neq 0$ jusqu'à  un point, où tout le reste de la matrice est $0$. Il faut s'appuyer sur les exercices \ref{item:17} et \ref{item:16}.}

\item Une matrice symétrique $A \in \R^{n \times n}$ est définie négative, si et seulement si $\det(B_k) \neq 0$ et $\det(B_k) = (-1)^k|\det(B_k)|$ pour tout $k$. 
\item Une matrice symétrique $A \in \R^{n \times n}$ est semi-définie négative, si et seulement si $\det(B_K) = (-1)^{|K|}|\det(B_K)|$ pour tout $K\subseteq \{1,.\dots,n\}$.  
\item Montrer la partie \eqref{eq:13} du théorème~\ref{thr:19}. 
\item 
Soit $A \in \R^{n \times n}$ une matrice symétrique avec les valeurs propres $\lambda_1 \geq \dots \geq \lambda_n$. 
Soit $B_K$ une matrice comme décrite en dessus o\`u $|K| = k$ avec les valeurs propres  $\mu_1 \geq \dots  \geq \mu_{n-k}$. Pour $1 \leq i \leq k$, alors 
\begin{displaymath}
  \lambda_i \geq  \mu_i \geq  \lambda_{i+k}.
\end{displaymath}
 
\end{enumerate}








\begin{definition}
  \label{def:22}
  Une matrice hermitienne $A \in \C^{n \times n}$ est 
  \begin{itemize}
  \item définie positive, si $x^TA\overline{x}>0$ pour tout $x \in \C^n \setminus  \{0\},$ 
  \item définie négative, si $x^TA\overline{x}<0$ pour tout $x \in \C^n \setminus  \{0\},$ 
  \item semi-définie positive, si $x^TA\overline{x}\geq 0$ pour tout $x \in \C^n$,
  \item semi-définie négative, si $x^TA\overline{x}\leq 0$ pour tout $x \in \C^n$.  
  \end{itemize}
\end{definition}


Le théorème~\ref{thr:15} trouve son analogue comme suivant. La démonstration est un exercice. 

\begin{theorem}
\label{thr:20}
  Une matrice hermitienne $A \in \C^{n \times n}$ est 
  \begin{enumerate}
  \item définie positive,
    si et seulement si toutes ses valeurs propres sont (strictement) positives. 
  \item définie négative,  si et seulement si toutes ses valeurs propres sont (strictement) négatives. 
  \item semi-définie positive, si et seulement si toutes ses valeurs propres sont non-négatives (donc positives ou zéro).  
  \item semi-définie négative, si et seulement si toutes ses valeurs propres sont non-positives (négatives ou zéro).  
  \end{enumerate}
\end{theorem}


\section{La décomposition en valeurs singulières}
\label{sec:la-decomposition-en}


On commence avec un théorème qui décrit la décomposition en valeurs singulières et montre qu'elle existe. 


\begin{theorem}
  \label{thr:21}
  Une matrice $A \in \C^{m \times n}$ peut être décomposée comme 
  \begin{displaymath}
    A = P\cdot D \cdot Q
  \end{displaymath}
où $P \in \C^{m \times m}$ et $Q \in \C^{n \times n}$ sont unitaires et $D \in \R_{\geq 0}^{m \times n}$ est une matrice diagonale. Si $A$ est réelle, $P$ et $Q$ sont réelles. 
\end{theorem}


\begin{proof}
  La matrice $A^* \cdot  A$ est hermitienne et semi-définie positive puisque
  \begin{displaymath}
    x^*A^*Ax = (Ax)^* (Ax) \geq 0. 
  \end{displaymath}
Alors les valeurs propres de $A^*A$ sont non-négatives ($\lambda_i\geq 0$). Soient $\sigma_1^2 \geq \sigma_2^2 \dots \geq \sigma_n^2\geq 0$ les valeurs propres et soit $\{u_1,\dots,u_n\}$ une base orthonormale  correspondante de vecteurs propres. La matrice $Q \in \C^{n \times n}$ est la matrice dont les lignes sont $u_1^*, \dots, u_n^*$. 

Soit $ r \in \N_0$ tel que $\sigma_r >0$ et $\sigma_{r+1} = 0$; on a $\sigma_1
\geq \ldots \geq \sigma_r > 0=\sigma_{r+1} = \ldots = 0=\sigma_n  $. Nous construisons les vecteurs 
\begin{displaymath}
  v_i = A \, u_i  / \sigma_i, \, 1 \leq i \leq r. 
\end{displaymath}
Les $v_i$ sont orthonormaux, parce que 
\begin{displaymath}
  \|v_i\|^2 = (A \, u_i)^* A u_i / \sigma_i^2 = 1
\end{displaymath}
et pour $1 \leq i\neq j \leq r$, 
\begin{displaymath}
	v_i^* v_j = \frac{u_i^* \cdot A^* \cdot A \cdot u_j}{\sigma_i \sigma_j} =
	\frac{\sigma_j^2}{\sigma_i \sigma_j} u_i^* u_j= 0. 
\end{displaymath}
Avec le procédé de Gram-Schmidt, nous complétons les $v_i$ tels que $\{v_1,\dots,v_m\}$ est une base orthonormale de $\C^m$. Les colonnes de la matrices $P$ sont alors $v_1,\dots,v_m$ dans cet ordre. La matrice $D \in \C^{m\times n}$ est la matrice diagonale dont les $r$ premières composantes sur la diagonale sont $\sigma_1,\dots,\sigma_r$ dans cet ordre. Avec ces matrices  $P,D$ et $Q$ nous avons 
\begin{displaymath}
  A = P\cdot D \cdot Q 
\end{displaymath}
ou de manière équivalente 
\begin{displaymath}
  P^* \cdot A \cdot Q^* = D,
\end{displaymath}

Nous montrons  ça en détail. Nous avons 
\begin{equation}
  \label{eq:39}
     (P^* \cdot A \cdot Q^*)_{ij} = v_i^* A u_j 
\end{equation}
et c'est égal à zéro si $j > r$, parce que dans ce cas $A u_j =0$ dès que $u_j^*A^*Au_j = 0$.
Si $i>r$ \eqref{eq:39} pas égale  a zéro implique $A u_j \neq 0$ et alors $j≤r$. Mais dans ce cas, par construction, $v_i$ est orthogonal à  $A u_j / σ_j$ et \eqref{eq:39}  est néanmoins zéro. 


Et  si $1 ≤i,j \leq r$, alors 
\begin{displaymath}
	u_i^* A^* A u_j / \sigma_i = \frac{\sigma_j^2}{\sigma_i}\,u_i^* u_j  =
  \begin{cases}
    \sigma_i & \text{si } i=j\\
    0 & \text{sinon}.  
  \end{cases}
\end{displaymath}
\end{proof}


\begin{definition}
  \label{def:24}
  En suivant la notation du théorème~\ref{thr:21}, 
  les nombres $\sigma_1,\dots,\sigma_r$ sont les \emph{valeurs singulières} de $A$. La factorisation $A = P\cdot D \cdot Q$ est une \emph{décomposition en valeurs singulières}. 
\end{definition}




\begin{example}
  \label{exe:18}
  Trouver une décomposition en valeurs singulières de 
  \begin{displaymath}
    A =
    \begin{pmatrix}
      0 & -1.6  & 0.6 \\
      0 & 1.2 & 0.8 \\
      0 & 0 & 0 \\
      0 & 0 & 0
    \end{pmatrix}. 
  \end{displaymath}
On commence avec 
\begin{displaymath}
  A^* \cdot A =
  \begin{pmatrix}
    0 & 0 & 0 \\
    0 & 4 & 0 \\
    0 & 0 & 1
  \end{pmatrix}
\end{displaymath}
On obtient $\sigma_1 = 2, \sigma_2 = 1$ et $\sigma_3 = 0$. Les valeurs singulières sont les $\sigma_i>0$, i.e., $\sigma_1=2$ et $\sigma_2=1$. On calcule les vecteurs propres correspondant à $\sigma_1 = 2, \sigma_2 = 1$ et $\sigma_3 = 0$. La matrice $Q$ est 
\begin{displaymath}
  Q =
  \begin{pmatrix}
    0 & 1 & 0 \\
    0 & 0 & 1  \\
    1 & 0 & 0  
  \end{pmatrix}
\end{displaymath}
et
\begin{displaymath}
v_1 = \frac{1}{2} \cdot   A 
  \begin{pmatrix}
    0\\1\\0
  \end{pmatrix} =
  \begin{pmatrix}
    -0.8 \\ 0.6 \\ 0 \\ 0
  \end{pmatrix}, 
v_2 =  A 
  \begin{pmatrix}
    0\\0\\1
  \end{pmatrix} =
  \begin{pmatrix}
    0.6 \\ 0.8 \\ 0 \\ 0
  \end{pmatrix}.
\end{displaymath}
On complète avec $v_3 = e_3$ et $v_4 = e_4$, alors 
\begin{displaymath}
P =   \begin{pmatrix}-0.8 & 0.6 & 0 & 0\\0.6 & 0.8 & 0 & 0\\0 & 0 & 1 & 0\\0 & 0 & 0 & 1\end{pmatrix}
\end{displaymath}
et 
\begin{displaymath}
  \begin{pmatrix}-0.8 & 0.6 & 0 & 0\\0.6 & 0.8 & 0 & 0\\0 & 0 & 1 & 0\\0 & 0 & 0 & 1\end{pmatrix} \cdot
  \begin{pmatrix}
    2 & 0 & 0 \\
    0 & 1 & 0 \\
    0 & 0 & 0 \\
    0 & 0 & 0
  \end{pmatrix}
\cdot
\begin{pmatrix}
  0 & 1 & 0\\0 & 0 & 1\\1 & 0 & 0 
\end{pmatrix} = A. 
\end{displaymath}

\end{example}

\begin{definition}
  La \emph{pseudo inverse} d'une matrice 
  \begin{displaymath}
    D =
    \begin{pmatrix}
      \sigma_1 \\
      & \sigma_2 \\
      & & \ddots \\
      & & & \sigma_r \\
      & & & & 0 \\
      & & & & & \ddots  \\
      & & & & & & 0  \\      
    \end{pmatrix}
    \in \R^{m \times n}
  \end{displaymath}
où $\sigma_i \in \R_{>0}$ est 
\begin{displaymath}
  D^+ =  \begin{pmatrix}
      \sigma_1^{-1} \\
      & \sigma_2^{-1} \\
      & & \ddots \\
      & & & \sigma_r^{-1} \\
      & & & & 0 \\
      & & & & & \ddots  \\
      & & & & & & 0  \\      
    \end{pmatrix}
    \in \R^{n \times m}
\end{displaymath}
Toutes les composantes qui ne sont pas décrites sont zéro. 
La \emph{pseudo inverse} d'une matrice $A \in \C^{m \times n}$ avec une décomposition en valeurs singulières $A = P \cdot D \cdot Q$ est 
\begin{displaymath}
  A^+ = Q^* D^+ P^*. 
\end{displaymath}

\end{definition}


\begin{example}
  La pseudo inverse de la matrice $A$ d'exemple~\ref{exe:18} est 
  \begin{displaymath}
    A^+ = 
\begin{pmatrix}0 & 0 & 1\\1 & 0 & 0\\0 & 1 & 0\end{pmatrix} \cdot 
\begin{pmatrix}0.5 & 0 & 0 & 0\\0 & 1 & 0 & 0\\0 & 0 & 0 & 0\end{pmatrix}
\cdot
\begin{pmatrix}-0.8 & 0.6 & 0 & 0\\0.6 & 0.8 & 0 & 0\\0 & 0 & 1 & 0\\0 & 0 & 0 & 1\end{pmatrix} 
  \end{displaymath}
\end{example}

Pourquoi est-ce que nous parlons de \emph{la} pseudo inverse? Parce qu'elle est unique. 

\begin{theorem}
  \label{thr:22}
  Soit  $A \in ℂ^{m \times n}$, alors il existe au plus une seule matrice $X \in ℂ^{n \times m}$  telle que les quatre conditions de \emph{Penrose} sont satisfaites: 
  \begin{enumerate}[i)] 
  \item $AXA = A$ \label{pen1}
  \item $(AX)^* = AX$ \label{pen2}
  \item $XAX = X$ \label{pen3}
  \item $(XA)^* = XA$. \label{pen4}
  \end{enumerate}
\end{theorem}
\begin{proof}
  Soient $X$ et $Y$ deux matrices satisfaisant \ref{pen1}-\ref{pen4}. Alors
  \begin{eqnarray*}
    X & = & XAX \\
     & = & XAYAX \\
     & = & XAYAYAYAX\\
    & = & (XA)^*(YA)^*Y(AY)^*(AX)^*\\
    & = & A^*X^*A^*Y^*YY^*A^*X^*A^* \\
    & = & (AXA)^*Y^*YY^*(AXA)^* \\
    & = & A^* Y^* Y Y^* A^* \\
    & = & (YA)^* Y (AY)^* \\
    & = & YAYAY \\
    & = & YAY \\
    & = & Y. 
  \end{eqnarray*}
\end{proof}

\begin{theorem}
  \label{thr:23}
  La pseudo inverse d'une matrice $A \in \C^{m \times n}$ satisfait les conditions \ref{pen1}-\ref{pen4}. 
\end{theorem}

\begin{proof}
  Soit $A = PDQ$ une décomposition en valeurs singulières et $A^+ = Q^*D^+P^*$.  Il est facile de voir que $D^+$ satisfait les conditions \ref{pen1}-\ref{pen4} relatives à $D$. Les conditions sont aussi vite montrées pour $A$ et $A^+$. Par exemple \ref{pen1} est montrée comme suit : 
  \begin{eqnarray*}
    A A^+A & = & PDQQ^*D^+P^*PDQ \\
           & = & PDD^+DQ \\
           & = & PDQ\\
           & = & A. 
  \end{eqnarray*}
Il est un exercice de vérifier les  conditions \ref{pen2}-\ref{pen4}.
\end{proof}


\section{Encore les systèmes d'équations }
\label{sec:encore-les-systemes}
Nous considérons encore une fois un système 
\begin{equation}
  \label{eq:14}
  Ax = b, 
\end{equation}
où $A \in \C^{m \times n}$ et $b \in \C^m$. 

\begin{definition}
  \label{def:25}
  La \emph{solution minimale} de \eqref{eq:14} est 
  la solution du problème des moindres carrés
  \begin{displaymath}
    \min_{x \in \C^n} \|Ax - b\|^2 
  \end{displaymath}
    correspondant avec norme $\|x\|$ minimale.
\end{definition}



\begin{theorem}
  \label{thr:24}
  La solution minimale de \eqref{eq:14} est 
  \begin{displaymath}
    x = A^+ b,
  \end{displaymath}
où $A^+$ est la pseudo inverse de $A$. 
\end{theorem}

\begin{proof}
Tout d'abord on remarque que pour $B\in \mathbb{C}^{n \times n}$ unitaire et $ y\in \mathbb{C}^n, \|y\|^2=y^T\overline {y}=y^TB^T\overline { B}\overline { y }=\|By\|^2$. Ainsi on a 
  \begin{eqnarray*}
    \min_{x \in \C^n} \|Ax - b\| & = &      \min_{x \in \C^n} \|PDQx - b\| =     \min_{x \in \C^n} \|P^* (PDQx - b)\|        \\
     & = &     \min_{x \in \C^n} \|DQx - P^*b\| =     \min_{y \in \C^n} \|Dy - P^*b \| \\
      & = &       \min_{y \in \C^n} \|Dy - c \|
  \end{eqnarray*}
où $c  = P^*b$. Dès lors on peut facilement vérifier que $y$ est une solution minimale de $Dy=c \Leftrightarrow  Q^*y$ est une solution minimale de $Ax=b$. Mais comme les solutions optimales de $Dy=c$  sont les $y \in \C^n$ tels que $y_i = c_i / \sigma_i $ pour $1 \leq i \leq r$ et $y_{r+1} \dots y_n$ sont arbitraires alors la solution où $y_{r+1} =\dots= y_n=0$ est celle de norme minimale. Elle est donnée par
\begin{displaymath}
  y = D^+ c. 
\end{displaymath}
La solution minimale de \eqref{eq:14} est alors 
\begin{displaymath}
  x = Q^*y = Q^* D^+ P^* b = A^+b. 
\end{displaymath}
\end{proof}

\begin{example}
  \label{exe:19}
  Trouver la solution minimale du système 
  \begin{displaymath}
     \begin{pmatrix}
      0 & -1.6  & 0.6 \\
      0 & 1.2 & 0.8 \\
      0 & 0 & 0 \\
      0 & 0 & 0
    \end{pmatrix} x =
    \begin{pmatrix}
      5\\7\\3\\-2
    \end{pmatrix}.
  \end{displaymath}

La pseudo-inverse  de la matrice ci-dessus est 
\begin{displaymath}
  A^+ = \begin{pmatrix}0 & 0 & 0 & 0\\-0.4 & 0.3 & 0 & 0\\0.6 & 0.8 & 0 & 0\end{pmatrix}
\end{displaymath}
et 
\begin{displaymath}
  \begin{pmatrix}0 & 0 & 0 & 0\\-0.4 & 0.3 & 0 & 0\\0.6 & 0.8 & 0 & 0\end{pmatrix} \cdot
  \begin{pmatrix}
    5\\7\\3\\-2
  \end{pmatrix}
 =
 \begin{pmatrix}
   0\\0.1\\8.6
 \end{pmatrix}. 
\end{displaymath}
\end{example}





\section{Le meilleur sous-espace approximatif} 
\label{sec:le-meilleur-sous}

Nous nous occupons du problème suivant. Soient $a_1,\dots,a_m \in \R^n$ des vecteurs et $1 \leq k \leq n$, trouver un sous-espace $H \subseteq \R^n$ de dimension $k$ tel que 
\begin{displaymath}
  \sum_i d(a_i,H)^2 
\end{displaymath}
soit minimale. Ici $d(a_i,H)$ est la \emph{distance} de $a_i$ a $H$. Si $H = \spa\{u_1,\dots,u_k\}$ où $\{u_1,\dots,u_k\}$  est une base orthonormale de $H$, alors $a_i = \sum_{j=1}^k \pscal{a_i,u_j} u_j + d_i$ où $d_i = a_i - \sum_{j=1}^k \pscal{a_i,u_j} u_j$ est orthogonal à $u_1,\dots,u_k$ et alors à $H$.  Avec le théorème de Pythagore (Proposition~\ref{prop:4}), on a
\begin{displaymath}
  d(a_i,H)^2 + \sum_{j=1}^k \pscal{a_i,u_j}^2 = \|a_i\|^2. 
\end{displaymath}
Le sous-espace $H$ de dimension $k$ qui minimise $\sum_i d(a_i,H)^2$ est alors celui qui maximise 
\begin{displaymath}
\sum_i  \sum_{j=1}^k \pscal{a_i,u_j}^2 = \sum_{j=1}^k \|A u_j\|^2 = \sum_{j=1}^k u_j^T A^TAu_j.
\end{displaymath}

Pour $k=1$, nous connaissons déjà une manière de résoudre ce problème. Il faut résoudre 
\begin{displaymath}
\max_{u \in S^{n-1}}   u^TA^TAu .
\end{displaymath}
La matrice $A^TA$ est symétrique, alors on peut la factoriser comme 
\begin{equation}
  \label{eq:17}
  A^TA = U
  \begin{pmatrix}
    \lambda_1 \\
    & \ddots \\
    & & \lambda_n
  \end{pmatrix} U^T
\end{equation}
où $U = \left(u_1,\dots,u_n\right) \in \R^{n \times n}$ est orthogonale. Nous pouvons supposer que les $\lambda_i$ sont ordonnés comme $\lambda_1 \geq \lambda_2 \geq \cdots \geq \lambda_n \geq 0$. Les valeurs propres sont non négatives dès que $A^TA$ est semi-définie positive. Selon Théorème~\ref{thr:18} la solution est $H = \spa\{u_1\}$. 

\medskip 
La généralisation suivante du Théorème~\ref{thr:18} est un exercice. 

\begin{theorem}
\label{thr:25}
  Soit $A \in \R^{n\times n}$ une matrice symétrique et $f(x) = x^TAx$ la forme quadratique correspondante à $A$.
Soit 
\begin{displaymath}
  A = U \cdot
  \begin{pmatrix}
    \lambda_1\\
    & \ddots \\
    & & \lambda_n
  \end{pmatrix}
U^T
\end{displaymath}
une factorisation de $A$ telle que $U = (u_1,\dots,u_n) \in \R^{n \times n }$ est orthogonale et $\lambda_1 \geq \cdots \geq \lambda_n$. 
Pour $1 \leq \ell <n$ on a
  \begin{equation}
\label{eq:15}
    \max_{\substack{x \in S^{n-1} \\ x \perp u_1, \dots, x\perp u_\ell}} f(x) = \lambda_{\ell +1}  = \min_{\substack{x \in S^{n-1} \\ x \perp u_{\ell+2}, \dots, x\perp u_n}} f(x)
  \end{equation}
et $u_{\ell+1}$ est une solution optimale. 
\end{theorem}



Maintenant, nous pouvons résoudre le problème central 
\begin{equation}
  \label{eq:16}
  \min_{\substack{H \trianglelefteq \R^n \\ \dim(H) = k}} \sum_{i=1}^m d(a_i,H)^2. 
\end{equation}

\begin{theorem}
  \label{thr:26}
  Soient $a_1,\dots,a_m \in \R^n$,
$
    A =\left(
      a_1,\cdots ,a_m\right)^T
 $
  et $u_1,\dots,u_k$  les premières colonnes de la matrice orthogonale $U \in \R^{n \times n}$ de la factorisation~\eqref{eq:17}. Le sous-espace $H = \spa\{u_1,\dots,u_k\}$ est une solution du problème~\eqref{eq:16}. 
\end{theorem}


\begin{proof}
Pour $k=1$ nous avons déjà montré l'assertion. Soit $k \geq 2$ et $W \trianglelefteq \R^n$ une solution optimale du problème~\eqref{eq:16} et soit $w_1,\dots,w_k$ une base orthonormale de $W$. 
Nous pouvons supposer $w_k \perp \spa\{u_1,\dots,u_{k-1}\}$, (voir Exercice~\ref{item:3}). 

Par induction, nous avons 
\begin{displaymath}
  \sum_{j=1}^{k-1} w_j^TA^TAw_j \leq  \sum_{j=1}^{k-1} u_j^TA^TAu_j.
\end{displaymath}
Dès que   
\begin{displaymath}
  \max_{\substack{x \in S^{n-1} \\ x \perp \spa\{u_1,\dots,u_{k-1}\}}} x^TA^TAx 
\end{displaymath}
est atteint à $u_k$ nous avons 
\begin{displaymath}
  w_k^TA^TAw_k \leq u_k^TA^TAu_k
\end{displaymath}
et alors
\begin{displaymath}
  \sum_{j=1}^{k} w_j^TA^TAw_j \leq  \sum_{j=1}^{k} u_j^TA^TAu_j.
\end{displaymath}

\end{proof}


\begin{definition}
  \label{def:26}
  Soit $A \in \R^{m \times n}$.
  \begin{enumerate}
 
  \item[1)] On définit la \emph{norme Frobenius} de $A$ comme le nombre 
  \begin{displaymath}
    \|A\|_F = \sqrt{\sum_{ij}a_{ij}^2}.
  \end{displaymath}
  
  \item[2)] Pour une norme vectorielle $\|\cdot\|$ définie positive sur $\Bbb R^n$, on définit la norme $\mnorm{\cdot}$ de $A$, appelée \emph{norme matricielle subordonnée} à $\|\cdot\|$, comme le nombre 
   \begin{displaymath}
    	\mnorm{A} = \sup_{v \neq 0} \frac{\|Av\|}{\|v\|} = \sup_{\|v\| = 1} \|Av\|.
  \end{displaymath}
  
   \end{enumerate}
\end{definition}

\begin{remark}
	La norme Frobenius n'est pas une norme matricielle subordonnée.
\end{remark}

\begin{example}
	Soit $A \in \R^{m \times n}$. On considère la norme euclidienne standard (aussi appelée norme 2) sur $\Bbb R^n$ : 
	$$\| v \|_2 = \sqrt{\sum\nolimits_{i=1}^n (v_i)^2}.$$
	Alors la norme matricielle subordonnée à $\|\cdot\|_2$ correspond à
	\begin{displaymath}
    	\mnorm{A}_2 = \sup_{v \neq 0} \frac{\|Av\|_2}{\|v\|_2} = \sup_{\|v\|_2 = 1} \|Av\|_2 = \sup_{\|v\|_2 = 1} \sqrt{v^T A^T A v} = \sqrt{\lambda_1},
  	\end{displaymath}
  	où $\lambda_1$ est la plus grande valeur propre de $A^T A$ (ou de manière équivalente, $\sqrt{\lambda_1}$ est la plus grande valeur singulière de $A$).
\end{example}

\begin{exercise}
	Soient $A \in \R^{m \times n}$, $\|\cdot\|$ une norme définie positive sur $\Bbb R^n$, $\mnorm{\cdot}$ la norme matricielle subordonnée à $\|\cdot\|$. Alors pour tout $v \in \Bbb R^n$, on a : 
	\begin{displaymath}
		\|Av\| \leq \mnorm{A} \cdot \|v\|.
	\end{displaymath}
\end{exercise}

\begin{definition}
  \label{def:27}
  Soit $A \in K^{n \times n}$. La \emph{trace} de $A$ est la somme de ses coefficients diagonaux, $\Tr(A) = \sum_{i=1}^n a_{ii}$. 
\end{definition}

\begin{lemma}
  \label{lem:11}
  Pour $A,B \in K^{n \times n}$ $\Tr(AB) = \Tr(BA)$. 
\end{lemma}


\begin{lemma}
  \label{lem:10}
  Pour $A \in \R^{m \times n}$, $\|A\|_F^2 = \sum_{i=1}^r \sigma_i^2$ où $\sigma_1, \dots, \sigma_r$ sont les valeurs singulières de $A$. 
\end{lemma}

\begin{proof}
  On a $\|A\|_F^2 = \Tr(A^TA) = \Tr(U \cdot \diag(\sigma_1^2,\dots,\sigma_n^2) \cdot  U^T)$ où $U \in \R^{n \times n}$ est orthogonale. Alors 
  \begin{displaymath}
    \|A\|_F^2 = \Tr(\diag(\sigma_1^2,\dots,\sigma_n^2)) =  \sum_{i=1}^r \sigma_i^2
  \end{displaymath}
\end{proof}




Maintenant nous allons résoudre le problème suivant. Étant donnés $A \in \R^{m \times n}$ et $k \in \N$, trouver une matrice $B \in \R^{m \times n}$ de $\rank(B) \leq  k$ tel que 
\begin{displaymath}
  \|A - B\|_F
\end{displaymath}
soit minimale. 

Si $A = P \cdot \diag(\sigma_1,\dots,\sigma_r,0,\dots 0) \cdot Q$ est une décomposition en valeurs singulières où les colonnes de $P$ sont $v_1,\dots,v_m$ et les lignes de $Q$ sont $u_1^T,\dots,u_n^T$ on peut écrire
\begin{equation}
  \label{eq:18}
  A = \sum_{i=1}^r \sigma_i v_iu_i^T
\end{equation}
et on dénote la somme des premiers $k$ termes comme 
\begin{displaymath}
  A_k = \sum_{i=1}^k \sigma_i v_iu_i^T
\end{displaymath}
Le rang de $A_k$ est au plus $k$. 

\begin{lemma}
  \label{lem:12}
  Les lignes de $A_k$ sont les projections des lignes de $A$ dans le sous-espace $V_k = \spa\{u_1,\dots,u_k\}$. 
\end{lemma}

\begin{proof}
  Soit $a^T$ une ligne de $A$. La projection de $a$ dans le sous-espace $\spa\{u_1,\dots,u_k\}$ est 
  \begin{displaymath}
    \sum_{i=1}^k a^Tu_i \cdot u^T_i.
  \end{displaymath}
Alors les projections des lignes de $A$ dans le sous-espace $V_k$ sont données par $\sum_{i=1}^k A u_i u_i^T = \sum_{i=1}^k \sigma_i v_i u_i^T = A_k$.  
\end{proof}

\begin{theorem}
  \label{thr:27}
  Pour une matrice $B \in \R^{m \times n}$ de rang plus petit ou égal à $k$, on a 
  \begin{displaymath}
    \|A - A_k\|_F \leq \|A - B\|_F.
  \end{displaymath}
\end{theorem}

\begin{proof}
On dénote les lignes de $A$ par $a_1^T,\dots,a_m^T$ et soit $B$ une matrice de rang au plus $k$. Les lignes de $B$ sont dénotées comme $b_1^T,\dots,b_m^T$. Soit $H = span\{b_1,\dots,b_m\}$. La dimension de $H$ est $\rank(B) \leq k$. On a 
\begin{displaymath}
  \|A - B\|_F^2 = \sum_{i=1}^m \|a_i - b_i\|^2 \geq \sum_{i=1}^m d(a_i,H)^2.
\end{displaymath}
Soit $\wt{H} = \spa\{u_1,\dots,u_k\}$. Nous avons démontré que 
\begin{enumerate}[i)]
\item $\wt{H}$ est le meilleur sous-espace approximatif des lignes de $A$ alors $\sum_{i=1}^m d(a_i,H)^2 \geq  \sum_{i=1}^m d(a_i,\wt{H})^2$ et 
\item Les lignes de $A_k$ sont les projections des lignes de $A$ dans $\wt{H}$. 
\end{enumerate}
En dénotant les lignes de $A_k$ par $\wt{a}_1^T,\dots,\wt{a_m}^T$, alors 
\begin{displaymath}
  \|A - B\|_F^2 \geq  \sum_{i=1}^m d(a_i,\wt{H})^2 = \sum_{i=1}^m \|a_i - \wt{a}_i\|^2 = \|A - A_k\|_F^2. 
\end{displaymath}
\end{proof}

\subsection*{Exercices}

\begin{enumerate}
\item Est-ce que la décomposition en valeurs singulières est unique? Est-ce que les valeurs singulières sont uniques?  \label{item:2}
\item  Dans la démonstration du théorème~\ref{thr:21}, montrer que $\rank(A) = r$. 
\item Démontrer que la pseudo-inverse satisfait les conditions \ref{pen2}-\ref{pen4}. 
\item Si $Ax = b$ a plusieurs solutions, il existe une solution unique avec une norme minimale. 
\item Montrer Théorème~\ref{thr:25}. 
\item Soient  $G,H \subseteq \R^n$ des sous-espaces de $\R^n$ 
et $k=\dim(G) > \dim(H)$. Montrer que $G$ possède une base orthonormale $w_1,\dots,w_k$ telle que $w_k \perp H$. \label{item:3}  
\end{enumerate}






%%% Local Variables:
%%% mode: latex
%%% TeX-master: "notes"
%%% End:

%%%%%%%%%%%%%%%%%%%%%%%%%%%%%%%%%%%%%%%%%%%

\documentclass[11pt, a4paper, oneside]{article}
\usepackage[utf8]{inputenc}
\usepackage[T1]{fontenc}
\usepackage[francais]{babel}
\usepackage{lmodern}
\usepackage[boldsans]{concmath}
\usepackage{utf8math}
\usepackage{enumerate} 

\usepackage{amsmath}
\usepackage{amssymb}
\usepackage{amsthm}
\usepackage{mathtools}
\usepackage{comment}
\usepackage{faktor}
\usepackage[dvipsnames]{xcolor}
\newcommand{\smat}[1]{ \big(\begin{smallmatrix} #1 \end{smallmatrix}\big)}

\newcommand{\indice}[1]{{\scriptsize \color{RubineRed} {#1}}}

%%%%%%%%%%%%%%%%%%%%%%%%%%%%%%%%%%%%%%%%%%%

\begin{document}
\title{Solutions Examen Algèbre Linéaire Avancée 2 2021}
\maketitle

\noindent 
\textbf{Attention:} Les solutions présentées sont très détaillées, les commentaires en \indice{ petits caractères} ne sont pas forcément nécessaires dans un vrai examen et servent à faciliter la compréhension si certaines étapes du raisonnement ne sont pas claires

\subsection*{Question 23}
\begin{enumerate}[i)] 
\item 
on veut montrer que $\det(A) = \prod_{i=1}^r \lambda_i^{m_i}$ \\
{\tiny par définition on a $p_A(x) = \det(A-xI)$ et donc une évaluation en $x=0$} donne directement $\alpha_0 = \det(A-0\cdot I) = \det(A)$ si on écrit $p_A(x) = \sum_{j=0}^n \alpha_jx^j$ \\
{\tiny de l'autre côté par supposition} le polynôme caractéristique se factorise de la manière suivante $p_A(x) = \alpha_n \prod_{i=1}^r (x-\lambda_i)^{m_i}$ \\
il reste à déterminer le coefficient $\alpha_n$ du polynôme caractéristique, par la formule de Leibniz on peut écrire $p_A(x) = \sum_{\sigma \in S_n} q_{\sigma}(x)$ pour des polynômes $q_{\sigma}(x) = \text{sgn}(\sigma)\prod_{i=1}^n(A-xI)_{i,\sigma(i)}$ \\
on remarque que le degré de $q_{\sigma}(x)$ est égal au nombre de points fixés par la permutation $\sigma$ \\
ainsi seulement la permutation $\sigma = id$ contribue au coefficient $\alpha_n$ de $p_A(x)$ \\
comme $q_{id}(x) = \prod_{i=1}^n(a_{ii}-x)$ on obtient directement que $\alpha_n = (-1)^n$ \\
on a donc $p_A(0) = (-1)^n \prod_{i=1}^r (0-\lambda_i)^{m_i} = \prod_{i=1}^r \lambda_i^{m_i}$ et {\tiny en utilisant le raisonnement tout en haut} on arrive à conclure que $\det(A) = \prod_{i=1}^r \lambda_i^{m_i}$ \\ \textcolor{blue}{4 points en total \\ on enlève 2 points si l'étudiant suppose que la matrice $A$ et diagonalisable, on enlève 2 points si le signe est mal placé dans une des expressions, on enlève 1 point pour tout autre erreur dans la formule de Leibniz}

\item 
on veut montrer que $\alpha_{n-1} = (-1)^{n-1}\text{Tr}(A)$ \\
{\tiny on observe comme avant que} seulement la permutation $\sigma = id$ contribue au coefficient $\alpha_{n-1}$ de $p_A(x)$ \\
comme on a $q_{id}(x) = \prod_{i=1}^n(a_{ii}-x)$ on voit que $$\alpha_{n-1} = (-1)^{n-1}a_{11} + \ldots + (-1)^{n-1}a_{nn} = (-1)^{n-1}\text{Tr}(A)$$
ce qui termine cette partie \\
\textcolor{blue}{3 points en total \\ on enlève entre 0.5 et 1 points pour des erreurs dans la formule de Leibniz, on enlève entre 0.5 et 1 points si la restriction à la permutation $\sigma = id$ n'est pas justifée, si l'étudiant écrit simplement la formule de Leibniz avec la justification sur la restriction à $\sigma = id$ mais sans conclure l'étudiant obtient 2 points}

\item 
on veut montrer que $\text{Tr}(A) = \sum_{i=1}^r m_i\lambda_i$ \\
{\tiny dans le point ii) on a montré que $\alpha_{n-1} = (-1)^{n-1}\text{Tr}(A)$,} cherchons maintenant le coefficient $\alpha_{n-1}$ dans la forme suivante du polynôme caractéristique $p_A(x) = (-1)^n \prod_{i=1}^r (x-\lambda_i)^{m_i}$ \\
on obtient
\begin{align*}
\alpha_{n-1} &= \underbrace{(-1)^{n+1}\lambda_1 + \ldots + (-1)^{n+1}\lambda_1}_{m_1 \text{fois}} + \ldots + \underbrace{(-1)^{n+1}\lambda_r + \ldots + (-1)^{n+1}\lambda_r}_{m_r \text{fois}} \\
&= (-1)^{n-1} \sum_{i=1}^r m_i\lambda_i
\end{align*}
en utilisant le résultat du point ii) on conclut que $\text{Tr}(A) = \sum_{i=1}^r m_i\lambda_i$ \\ 
\textcolor{blue}{3 points en total \\ on enlève 2 points si l'étudiant suppose que $A$ est diagonalisable, on enlève 1 point pour des erreurs de signe} 
\end{enumerate}


\subsection*{Question 26}
{\tiny toutes les normes sur l'espace vectoriel $\mathbb{C}^{n \times n}$ sont équivalentes (car il s'agit d'un espace vectoriel de dimension finie) mais pour fixer nos idées on considère la convergence par rapport à la norme $\lVert \cdot \rVert_2$}
\subsubsection*{a)}
si $\rho(A) \geq 1$ alors on veut montrer que $A^n$ ne converge pas vers la matrice nulle \\
{\tiny on fait une preuve par contradiction,} supposons $\rho(A) \geq 1$ et $A^n$ converge vers la matrice 0 \\
prenons une valeur propre $\lambda$ de la matrice $A$ tel que $|\lambda| = \rho(A)$ et prenons aussi un vecteur propre unitaire $x \in \mathbb{C}^n$ associé à $\lambda$ \\
pour tout $n \in \mathbb{N}$ on a $$\lVert A^n \rVert_2 = \lVert A^n \rVert_2 \lVert x \rVert_2 \geq \lVert A^nx \rVert_2 = \lVert \lambda^n x \rVert_2 = |\lambda|^n \geq 1$$ {\tiny où la dernière inégalité est une conséquence des faits $|\lambda|\geq 1$ et $\lVert x \rVert_2 = 1$} \\
mais cela nous donne une contradiction avec la supposition que $A^n$ converge vers la matrice 0 car on a montré que $\lVert A^n \rVert_2$ est supérieur ou égal à 1 $\forall n \in \mathbb{N}$ \\
{\tiny on a donc prouvé que $A^n$ ne converge pas vers la matrice 0} \\
\textcolor{blue}{3 points en total \\ 1 point pour l'idée et le début de la preuve par contradiction, 1 point pour le calcul $\lVert A^n \rVert_2 \geq 1$ et 1 point pour conclure l'argument} 

\subsubsection*{b)}
soit $B=\lambda I+N$ où $\lambda \in \mathbb{C}$ et $N \in \mathbb{C}^{n \times n}$ est une matrice nilpotente \\
on veut montrer que si $|\lambda|<1$ alors $\lim_{n\rightarrow \infty}B^n = 0$ \\
{\tiny on remarque d'abord que comme $N$ est nilpotente il existe $k \in \mathbb{N}$ tel que $N^k = 0$} \\
{\tiny en plus comme $\lambda I$ est une matrice scalaire elle commute avec la matrice $N$,} on peut donc appliquer la formule du binôme de Newton \\
on suppose $n>k$ et on calcule
\begin{align*}
    B^n &= (\lambda I + N)^n \\
    &= \sum_{i=0}^n \begin{pmatrix} n \\ i \end{pmatrix} \lambda^{n-i}I \cdot N^i \\
    &= \sum_{i=0}^{k-1} \begin{pmatrix} n \\ i \end{pmatrix} \lambda^{n-i}I \cdot N^i + \underbrace{\sum_{i=k}^n \begin{pmatrix} n \\ i \end{pmatrix} \lambda^{n-i}I \cdot N^i}_{=0} \\
    &= \sum_{i=0}^{k-1} \begin{pmatrix} n \\ i \end{pmatrix} \lambda^{n-i}I \cdot N^i
\end{align*}
{\tiny un résultat d'Analyse nous donne que} $$\lim_{n \rightarrow \infty} \begin{pmatrix} n \\ i \end{pmatrix} \lambda^{n-i} = 0$$ pour $i \in \{0,\ldots,k-1\}$ fixé et $|\lambda|<1$ \\
{\tiny on donc montré que $B^n$ est égal à une somme finie dont} chaque terme converge vers 0 quand $n \rightarrow \infty$, cela implique $\lim_{n\rightarrow \infty}B^n = 0$ \\
\textcolor{blue}{4 points en total \\ 1 point pour l'utilisation de la formule du binôme de Newton, 2 points pour simplifier l'expression correctement à l'aide de la nilpotence de $N$ et 1 point pour conclure l'argument à l'aide de la limite $\lim_{n \rightarrow \infty} \begin{pmatrix} n \\ i \end{pmatrix} \lambda^{n-i}$} 

\subsubsection*{c)} 
on montre les deux direction du si et seulement si
\begin{itemize}
    \item [$\Longrightarrow$] pour montrer cette direction il suffit de montrer la contraposée: si $\rho \geq 1$ alors la suite $A^n$ ne converge pas vers la matrice 0 \\
    mais c'est exactement ce qu'on a montré dans le point a) de l'exercice \\
    {\tiny on a donc prouvé cette direction de l'équivalence}
    \item [$\Longleftarrow$] pour l'autre direction {\tiny on prend d'abord une valeur propre $\mu$ de $A$ associé au vecteur propre $y \in \mathbb{C}^n$, alors on obtient par un calcul direct la relation $A^ny = \mu^n y$} \\
    prenons maintenant un vecteur propre unitaire $x \in \mathbb{C}^n$ associé à la valeur propre $\lambda$ qui vérifie $|\lambda| = \rho(A)$, alors on a
    \begin{align*}
        0 &\leq \rho(A)^n \\
        &= |\lambda|^n \\
        &= \lVert \lambda^nx \rVert_2 \\
        &= \lVert A^nx \rVert_2 \\
        &\leq \lVert A^n \rVert_2 \lVert x \rVert_2 \\
        &= \lVert A^n \rVert_2
    \end{align*}
    comme on a $\lVert A^n \rVert_2 \rightarrow 0$ par supposition on voit que $\rho(A)^n \rightarrow 0$ par le théorème des deux gendarmes \\
    ainsi on a nécessairement $\rho(A)<1$, {\tiny ce qui termine la preuve}
\end{itemize}
\textcolor{blue}{3 points en total \\ 1 point pour la direction $\Longrightarrow$ de l'équivalence et 2 points pour l'autre direction (1 point pour le calcul $A^ny = \mu^n y$ et pour borner $\rho(A)^n$ supérieurement et 1 point pour conlure à l'aide du théorème des deux gendarmes)} 
    


\newpage
\subsection*{Question 27}
\subsubsection*{a)}
soit $H\subset\mathbb{R}^n$ un sous-espace vectoriel de $\mathbb{R}^n$ et soit $v\in \mathbb{R}^n$, soit $l\in H$ tel que $\langle v-l, h\rangle = 0$ pour tout $h \in H$ \\
\text
on veut montrer que $\lVert v-l \rVert \leq \lVert v \rVert$ \\
remarquons d'abord que qu'on a aussi $\langle v-l,l \rangle = 0$ car $l \in H$, on peut donc appliquer le théorème de Pythagore dans le calcul suivant
$$\lVert v \rVert^2 = \lVert v-l + l \rVert^2 = \lVert v-l \rVert^2 + \lVert l \rVert^2$$
{\tiny comme $\lVert \cdot \rVert$ est une norme} on voit que $\lVert l \rVert^2 \geq 0$ et par conséquent on a $\lVert v \rVert^2 \geq \lVert v-l \rVert^2$, {\tiny ainsi $\lVert v \rVert \geq \lVert v-l \rVert$} \\
\textcolor{blue}{3 points en total \\ on donne 0 points si $\langle v,l \rangle \geq 0$ est utilisé sans justification, on donne 0 points pour $\lVert v-l \rVert^2 = \langle v,v \rangle - \langle v,l \rangle$, on enlève un point si par exemple l'application de Pythagore n'est pas justifiée}

\subsubsection*{b)}
soit $A\in \mathbb{R}^{n \times n}$ de rang plein et soit $A=A^*R$ une factorisation selon la méthode de Gram-Schmidt {\tiny (les colonnes de $A^* \in \mathbb{R}^{n \times n}$ sont deux-à-deux orthogonaux et $R \in \mathbb{R}^{n \times n}$ est une matrice triangulaire supérieure avec que de 1 sur la diagonale)} \\
on veut montrer que $$\det(A^TA) = \prod_{i=1}^n \lVert a_i^* \rVert^2$$
où $a_i^*$ est la $i$-ème colonne de la matrice $A^*$ \\
on remarque d'abord qu'on a $\det(R)=1$ {\tiny car $R$ est une matrice triangulaire supérieure avec que de 1 sur la diagonale} \\
calculons
\begin{align*}
    \det(A^TA) &= \det((A^*R)^T(A^*R)) \\
    &= \det(R^TA^{*T}A^*R) \\
    &= \underbrace{\det(R^T)}_{=1}\underbrace{\det(R)}_{=1}\det(A^{*T}A^*) \\
    &= \det(A^{*T}A^*)
\end{align*}
regardons maintenant la matrice $A^{*T}A^*$, {\tiny on sait que les colonnes de $A^*$ sont deux-à-deux orthogonaux} \\
ainsi par définition du produit matriciel on voit que
$$(A^{*T}A^*)_{ij} = a_i^{*T}a_j^* = \begin{cases} \lVert a_i^* \rVert^2 & \text{si} \; i=j \\ 0 & \text{si} \; i\neq j \end{cases}$$
on voit donc que $A^{*T}A^* = \text{diag}(\lVert a_1^* \rVert^2, \ldots, \lVert a_n^* \rVert^2)$ et par conséquent $\det(A^{*T}A^*) = \prod_{i=1}^n \lVert a_i^* \rVert^2$, par le calcul qui précède on obtient
$$\det(A^TA) = \prod_{i=1}^n \lVert a_i^* \rVert^2$$
{\tiny ce qui termine cette partie} \\
\textcolor{blue}{4 points en total \\ on donne 2 points pour montrer $\det(A^TA) = \det(A^{*T}A^*)$ et 2 points pour la preuve de $\det(A^{*T}A^*) = \prod_{i=1}^n \lVert a_i^* \rVert^2$}

\subsubsection*{c)}
on veut montrer que $|\det(A)| \leq \prod_{i=1}^n \lVert a_i \rVert$ {\tiny où $a_i$ est la $i$-ième colonne de la matrice $A$} \\
{\tiny par le point b) on $\det(A^TA) = \prod_{i=1}^n \lVert a_i^* \rVert^2$} et on sait que $\det(A^TA) = \det(A^T)\det(A) = \det(A)^2$ \\
on obtient donc $|\det(A)| = \prod_{i=1}^n \lVert a_i^* \rVert$ \\
comme on a utilisé la méthode de Gram-Schmidt pour définir $A^*$ on a la relation $a_i^* = a_i -l$ où $l \in \text{span}\{a_1^*,\ldots,a_{i-1}^*\}$, en plus on a aussi $a_i^* \perp \text{span}\{a_1^*,\ldots,a_{i-1}^*\}$ \\
on peut donc poser $H = \text{span}\{a_1^*,\ldots,a_{i-1}^*\}$ et conlure par le point a) que $\lVert a_i^* \rVert = \lVert a_i-l \rVert \leq \lVert a_i \rVert$ \\
on en déduit que $$|\det(A)| = \prod_{i=1}^n \lVert a_i^* \rVert \leq \prod_{i=1}^n \lVert a_i \rVert$$
{\tiny ce qui termine la preuve} \\
\textcolor{blue}{3 points en total \\ on donne 1 point pour montrer $|\det(A)| = \prod_{i=1}^n \lVert a_i^* \rVert$ et 2 points pour $\lVert a_i^* \rVert \leq \lVert a_i \rVert$ en utilisant a) et après avoir verifié les conditions}

\end{document}
%%% Local Variables:
%%% mode: latex
%%% TeX-master: t
%%% End:

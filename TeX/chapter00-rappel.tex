\chapter{Rappel des notions  fondamentales}
\label{cha:noti-fond-et}

\noindent 
On se rappelle de la notion d'un espace vectoriel $V$ sur un corps $K$. Pendant notre cours, on travaille presque exclusivement avec des espaces vectoriels de dimension fini, $\dim(V) = n ∈ ℕ$. Dans ce cas il existe une base finie
\begin{displaymath}
  B = \{ v_1,\dots,v_n\}. 
\end{displaymath}
On considère cette base $B$ comme ordonnée. Alors il y a un premier élément de cet ensemble $B$  un deuxième etc.  


Tout élément $v ∈ V$ possède une représentation unique comme \emph{combinaison linéaire}
\begin{displaymath}
  v = α_1 v_1 + \cdots + α_n v_n, 
\end{displaymath}
où  $α_i ∈ K$ pour $i=1,\dots,n$. Le vecteur
\begin{displaymath}
  [v]_B =
  \begin{pmatrix}
    α_1\\ \vdots \\ α_n
  \end{pmatrix} ∈ Κ^n
\end{displaymath}
est appelé les \emph{coordonnées} de $v$ \emph{relatif à la base} $B$. Si $B'$ est une autre base de $V$ on peut obtenir  les \emph{coordonnées} de $v$ relatif à la base $ℬ'$ à l'aide de la \emph{ matrice de changement de base} $P_{B B'} ∈ K^{n ×n}$. Pour tout $v ∈ V$ on a
\begin{displaymath}
  [v]_{B'} =  P_{BB'} [v]_B. 
\end{displaymath}

\begin{example}
  \label{exe:b-21}
  Supposons  $\dim(V) = 3$ et $B = \{ v_1,v_2,v_3\}$ et $B' = \{ v'_1,v'_2,v'_3\}$ sont deux bases de $V$.  Les $v_i'$ sont représentées comme combinaisons linéaires des $v_i$ comme suivant.
    \begin{eqnarray*}
      v_1' & = &  2 v_1 + v_2 - v_3\\
      v_2' & = &   v_1 + 3v_2 + v_3 \\
      v_3' & = &   v_1 + v_2 -  v_3. 
    \end{eqnarray*}
    La matrice de changement de base $P_{B'B}$ est
    \begin{displaymath}
      P_{B'B} = \left(\begin{array}{rrr}
2 & 1 & 1 \\
1 & 3 & 1 \\
-1 & 1 & -1
\end{array}\right)
\end{displaymath}
On vérifie par exemple 
\begin{displaymath}
  [v_1']_{B'}  = e_1 \text{   et  } [v_1']_{B} = P_{B'B} ⋅ e_1. 
\end{displaymath}
La matrice de changement de base $P_{BB'}$ est
\begin{displaymath}
  P_{BB'} =  P_{B'B}^{-1} = \left(\begin{array}{rrr}
1 & -\frac{1}{2} & \frac{1}{2} \\
0 & \frac{1}{4} & \frac{1}{4} \\
-1 & \frac{3}{4} & -\frac{5}{4}
\end{array}\right).
\end{displaymath}
\end{example}
%
\noindent 
Comme  dans l'exemple~\ref{exe:b-21} pour $i ∈ \{1,\dots,n\}$  le vecteur $e_i ∈ K^n$ est
\begin{displaymath}
  e_i =
  \begin{pmatrix}
    0 \\ \vdots \\ 0 \\ 1 \\ 0 \\ \vdots \\1
  \end{pmatrix}
\end{displaymath}
où la $i$-ième composante est $1$ et les autres sont tous égaux à $0$.  


\section{Applications linéaires et matrices}
\label{sec:appl-line-et}
\noindent 
Soient $V$ et $W$ deux espaces vectoriels sur un corps $K$. Une \emph{application linéaire} est une fonction $f: V ⟶ W$ tel que
\begin{enumerate}[(l.1)] 
\item Pour tous $u,v ∈ V$ on a $f(u+v) = f(u) + f(v)$. 
\item Pour tout $u ∈V$ et $α ∈ K$ on a $f(α u) = α f(u)$. 
\end{enumerate}


\medskip
\noindent 
Si $B_1 = \{v_1,\dots,v_n\}$ est une base de $V$ et  $B_2 = \{w_1,\dots,w_m\}$ est une base de $W$ et si
\begin{displaymath}
  f(v_j) = a_{1j} w_1 + \cdots + a_{mj} w_m 
\end{displaymath}
alors pour tout $v ∈ V$ on vérifie facilement
\begin{displaymath}
  [f(v) ]_{B_2} =
  \begin{pmatrix}
    a_{11} & a_{12} & \cdots & a_{1n} \\
    \vdots &&&\vdots \\
     a_{m1} & a_{m2} & \cdots & a_{mn} \\
  \end{pmatrix}
  [v]_{B_1}.  
\end{displaymath}
% La matrice
<% \begin{equation}
%   \label{eq:19}
%   A_{B_1,B_2}^f =  \begin{pmatrix}
%     a_{11} & a_{12} & \cdots & a_{1n} \\
%     \vdots &&&\vdots \\
%      a_{m1} & a_{m2} & \cdots & a_{mn} \\
%   \end{pmatrix}
% \end{equation}
% est appelé la matrice de $f$ relatif 



\begin{example}
  \label{example-b2}
  Soient  $V$ et $W$ deux espaces vectoriels sur $ℤ_3$,  $B_1 = \{ v_1,v_2,v_3,v_4\}$ et $B_2 = \{ u_1,u_2,u_3\}$  bases de $V$ et $W$ respectivement. Soit $f : V ⟶ W$ une application linéaire  et les images des $v_i$ comme suivant.
  \begin{eqnarray*}
 f(v_ 1 ) & =&   1 \cdot u_ 1 + 2 \cdot u_ 2 + 2 \cdot u_ 3 + 2 \cdot u_ 4  \\ 
 f(v_ 2 ) & =&   2 \cdot u_ 1 + 1 \cdot u_ 2 + 1 \cdot u_ 3 + 1 \cdot u_ 4  \\ 
 f(v_ 3 ) & =&   0 \cdot u_ 1 + 2 \cdot u_ 2 + 0 \cdot u_ 3 + 2 \cdot u_ 4  \\ 
 f(v_ 4 ) & =&   1 \cdot u_ 1 + 2 \cdot u_ 2 + 2 \cdot u_ 3 + 2 \cdot u_ 4  \\ 
 f(v_ 5 ) & =&   2 \cdot u_ 1 + 2 \cdot u_ 2 + 1 \cdot u_ 3 + 2 \cdot u_ 4 
  \end{eqnarray*}
  Alors la matrice
  \begin{displaymath}
    A =
    \left(\begin{array}{rrrrr}
1 & 2 & 0 & 1 & 2 \\
2 & 1 & 2 & 2 & 2 \\
2 & 1 & 0 & 2 & 1 \\
2 & 1 & 2 & 2 & 2
\end{array}\right) ∈ ℤ_3^{4 ×5} 
 \end{displaymath}
 nous donne pour tout $v ∈ V$ 
 \begin{displaymath}
   [f(v)]_{B_2}  = A [v]_{B_1}. 
 \end{displaymath}
 Par exemple on observe 
 \begin{eqnarray*}
   f(v_1+ 2 v_3+ 1 v_5) &= &      1 ⋅  (1 \cdot u_ 1 + 2 \cdot u_ 2 + 2 \cdot u_ 3 + 2 \cdot u_ 4)  \\ 
                            & & + 2 ⋅  (0 \cdot u_ 1 + 2 \cdot u_ 2 + 0 \cdot u_ 3 + 2 \cdot u_ 4)  \\
                            & & + 1 ⋅  (2 \cdot u_ 1 + 2 \cdot u_ 2 + 1 \cdot u_ 3 + 2 \cdot u_ 4) \\ 
                            & = &      0 \cdot u_1 + 2 \cdot u_2 + 0 \cdot u_ 3 + 2 \cdot u_ 4 
 \end{eqnarray*}
 et
 \begin{eqnarray*}
   A ⋅ [ v_1+ 2 v_3]_{B_1} & = & A ⋅
                                 \begin{pmatrix}
                                   0 \\ 1 \\ 0 \\1 
                                 \end{pmatrix} \\
   & = &
         \begin{pmatrix}
           0 \\  2 \\ 0\\ 2
         \end{pmatrix}
 \end{eqnarray*}
\end{example}




\section{L'élimination de Gauss}
\label{sec:analys-gauss-elim}

Rappelons l'élimination de Gauss. Étant donné une matrice $A ∈ K^{m ×n}$, cette algorithme transforme $A$ en $A'$ en \emph{forme échelonée}. C'est à dire que:
\begin{enumerate}[i)] 
\item Si $A'$ possède $r$ lignes non nuls  ($≠ 0^T$), ces lignes  sont les premiers $r$ lignes de $A'$.
\item Il existe une suite $1≤ p_1 < p_2 < \cdots < p_k ≤ n$  les \emph{indices pivot} t.q. pour tout $i ∈ \{1,\dots,r\}$,
  $A'_{i,p_i} ≠0$ et   pour tout $1 ≤ j < p_i$ on a   $A'_{i,p_i} =0$.
\end{enumerate}


Le nombre  de zéros précédant le premier élément non nul \emph{pivot} de chaque ligne non zéro de $A'$ qui n'est pas égale a l augmente strictement à chaque ligne,     en effectuant des opérations élémentaires de lignes

\begin{algorithm}%[Élimination de Gauss]
  \label{alg:3}
  \caption{Élimination de Gauss}
\begin{tabbing}
\textbf{Entrée:} $A ∈ K^{m ×n}$ \\
\textbf{Sortie:} \= $A'∈ K^{m ×n}$ sous forme échelonnée telle qu'il existe une matrice inversible \\
\> $Q ∈ K^{m × m}$ vérifiant $Q⋅A = A'$. \\
\pushtabs

$A' := A$ \\
$i := 1$\\
\textbf{tant que} = ($i≤m$) \\
\> trouver le \emph{plus petit} $1 ≤ j ≤n$ tel qu'il existe $k≥i$ avec $a'_{kj} ≠ 0$ \\ 
\> Si un tel élément n'existe pas, alors \textbf{ arrêter} \\
\> échanger les lignes $i$ et $k$ dans $A'$ \\
\>  \textbf{pour} \= $k = i+1,\dots, m$ \\
\>            \> soustraire $(a'_{kj}/a'_{ij})$ fois la ligne $i$ de la ligne $k$ dans $A'$ \\
\> $i:=i+1$ \\
\poptabs
\textbf{retourner:}  $A'$ 
\end{tabbing}
\end{algorithm}

\noindent
On peut facilement prouver que l'algorithme est correcte. Tout d'abord, l'algorithme n'effectue que des opérations élémentaires sur les lignes de la forme
\begin{enumerate}[i)]
\item échanger deux lignes  \label{item:5}
\item additionner  un multiple d'une ligne à une \textbf{autre} ligne. \label{item:10}
\end{enumerate}
L'opération \ref{item:5}), si les lignes $i$ et $j$ sont échangées,  correspond à la multiplication
\begin{displaymath}
  A'  :=
  % \begin{pmatrix}
  %   e_1^T \\ \vdots \\ e^T_{i-1} \\ e^T_j \\ e^T_{i+1} \\ \vdots  \\ e^T_{j-1} \\ e^T_i \\ e^T_{j+1} \\ \vdots  \\ e^T_m
  % \end{pmatrix}
  T_{i,j}⋅ A', 
\end{displaymath}
où $T_{i,j} ∈ K^{m ×m}$  est obtenue à partir de la matrice identité $I ∈ K^{m ×m}$ en échangeant la $i$-ème et la $j$-ème ligne.

L'opération~\ref{item:10}), si on additionne $α$ fois la ligne $i$ sur la ligne $j$, correspond à la multiplication 
\begin{displaymath}
  A'  :=
  % \begin{pmatrix}
  %   e_1^T \\
  %   \vdots \\
  %   e_{j-1}^T \\
  %   α ⋅ e_i^T + e_j^T \\
  %   e_{j+1}^T \\
  %   \vdots \\
  %   e_m^T
  % \end{pmatrix}
L_{i,j}(α)  ⋅ A', 
\end{displaymath}
où $L_{i,j}(α) ∈ K^{m ×m}$ est obtenue à partir de la matrice identité $I ∈ K^{m ×m}$ en additionnant $α$ \emph{fois} la $i$-ème sur  la $j$-ème ligne.
\begin{example}
  \label{exe:21}
  En $ℚ^{4×4}$ on a
  \begin{displaymath}
    T_{1,3}  =
    \begin{pmatrix}
      0 & 0 & 1 & 0\\
      0 & 1 & 0 & 0\\
      1 & 0 & 0 & 0 \\
      0 & 0 & 0 & 1
    \end{pmatrix} \quad \text{  et } \quad
     L_{3,2}(-5)  =
    \begin{pmatrix}
      1& 0 & 0 & 0\\
      0 & 1 & -5 & 0\\
      0 & 0 & 1 & 0 \\
      0 & 0 & 0 & 1
    \end{pmatrix}
  \end{displaymath}
\end{example}
La matrice  résultante $A'$ peut être obtenue via
\begin{displaymath}
A' = Q ⋅ A
\end{displaymath}
avec une matrice $Q ∈ K^{m × m}$ non singulière.
En fait, si les opérations sont effectuées  en multipliant $A'$ avec  les matrices élémentaires $P_1 , \dots, P_μ$ à gauche, $Q$ est le produit
\begin{displaymath}
  Q = P_μ \cdots P_1. 
\end{displaymath}
Par ailleurs, nous avons l'invariant suivant.
\begin{quote}
Après chaque itération de la boucle \textbf{ tant que}, la matrice $H$ obtenue à partir des $j$ premières colonnes de $A'$ est sous forme échelonnée et les lignes $i, i+1, \dots, m$ de $H$ sont entièrement nulles, voir la figure~\ref{fig:9}.
\end{quote}

\begin{figure}
\centering
\includegraphics[height=5cm]{./Figures/Echelon.jpg}
\caption{Élimination de Gauss : la matrice $A'$ avant la $i$-ème itération de la boucle \textbf{ tant que}. \label{fig:9}}
\end{figure}
\noindent 
Combien d'opérations arithmétiques l'élimination de Gauss effectue-t-elle ? Soustraire un multiple d'une ligne à une autre ligne dans $A'$ peut se faire en $O(n)$ opérations arithmétique du corps $K$. Ainsi, le nombre total d'opérations effectuées dans la boucle \textbf{ tant que} est $O(m ⋅n)$. Il y a $O(m)$ itérations de la boucle \textbf{ tant que}. Au total, cela montre que l'élimination de Gauss effectue $O(m^2 ⋅n)$ opérations.



\begin{example}[Exemple~\ref{example-b2} continué] 
  On déroule l'algorithme de Gauss sur la matrice
  \begin{displaymath}
    A =
    \left(\begin{array}{rrrrr}
      1 & 2 & 0 & 1 & 2 \\
      2 & 1 & 2 & 2 & 2 \\
      2 & 1 & 0 & 2 & 1 \\
      2 & 1 & 2 & 2 & 2
    \end{array}\right) ∈ ℤ_3^{4 ×5}, 
\end{displaymath}
et en calcule la matrice $Q ∈  ℤ_3^{4 ×4}$ telle que  $Q ⋅A$ est en forme échelonnée. Au début $A' := A$ et $Q := I_4$.
On  effectue les opérations suivantes sur $A'$ et $Q$:
\begin{enumerate}[] 
\item $\ligne(2):= \ligne(2) - 2 ⋅ \ligne(1)$
\item $\ligne(3):= \ligne(3) - 2 ⋅ \ligne(1)$
\item $\ligne(4):= \ligne(4) - 2 ⋅ \ligne(1)$. 
\end{enumerate}
Après nous avons
\begin{displaymath}
  A ' =
  \left(\begin{array}{rrrrr}
1 & 2 & 0 & 1 & 2 \\
0 & 0 & 2 & 0 & 1 \\
0 & 0 & 0 & 0 & 0 \\
0 & 0 & 2 & 0 & 1
\end{array}\right)
\quad \text{ et } Q =
\left(\begin{array}{rrrr}
1 & 0 & 0 & 0 \\
1 & 1 & 0 & 0 \\
1 & 0 & 1 & 0 \\
1 & 0 & 0 & 1
\end{array}\right).
\end{displaymath}
Enfin, après l'opération suivante sur $A'$ et $Q$
\begin{enumerate}[] 
\item $\ligne(4):= \ligne(4) - 1 ⋅ \ligne(2)$
\end{enumerate}
on obtient
\begin{displaymath}
  A ' =
  \left(\begin{array}{rrrrr}
1 & 2 & 0 & 1 & 2 \\
0 & 0 & 2 & 0 & 1 \\
0 & 0 & 0 & 0 & 0 \\
0 & 0 & 0 & 0 & 0
\end{array}\right)
\quad \text{ et } Q =
\left(\begin{array}{rrrr}
1 & 0 & 0 & 0 \\
1 & 1 & 0 & 0 \\
1 & 0 & 1 & 0 \\
0 & 2 & 0 & 1
\end{array}\right)
.
\end{displaymath}
La matrice $A'$ est en forme échelonnée et on vérifie
\begin{displaymath}
  Q ⋅ A = A'.
\end{displaymath}
\end{example}

\section{L'image et le noyaux d'une application linéaire}
\label{sec:limage-et-le}
Soient encore $V$ et $W$ deux espaces vectoriels sur $K$ de dimension fini et $f : V ⟶ W$ une application linéaire. On se rappelle de la définition de l'\emph{image} de $f$
\begin{displaymath}
  \im(f) = \{ f(v) : v ∈ V\} ⊆ W. 
\end{displaymath}
L'image $\im(f)$ est un sous espace vectoriel de $W$. Le noyaux de $f$ est défini comme 
\begin{displaymath}
  \ker(f) = \{ v ∈ V : f(v) = 0\} ⊆ V. 
\end{displaymath}
le noyaux  $\ker(f)$ est un sous espace vectoriel de $V$. On se rappelle du théorème suivant du premier semestre~\cite{PMichel2026}.
\begin{theorem}
  \label{thr:29}
  Soient $V$ et $W$ deux espaces vectoriels sur $K$ de dimension fini et $f : V ⟶ W$ une application linéaire. Alors
  \begin{displaymath}
    \dim(V) = \dim(\im(f)) + \dim(\ker(f)). 
  \end{displaymath}
\end{theorem}
Comment calculer des bases des espaces vectoriels $\im(f)$ et $\ker(f)$? Si $A' ∈ K^{m × n}$  est en forme échelonnée  avec des indices de pivot $1≤ p_1 < p_2 < \cdots < p_k ≤ n$ et si  les colonnes de $A'$ sont $a'_1,\dots, a'_n ∈ Κ^m$, alors les vecteurs
\begin{equation}
  \label{eq:b-21}  
  a'_{p_1},\dots,a'_{p_k} \text{ sont linéairement indépendants.}
\end{equation}
Aussi pour tout $j ∈ \{1,\dots,n \} \setminus  \{p_1, \cdots, p_k \}$, le vecteur  $a'_j∈ K^m$ est une combinaison linéaire des colonnes $a'_{p_1},\dots,a'_{p_ℓ}$ où $ℓ$ est maximal tq. $j> p_ℓ$. 
\begin{displaymath}
  a'_j = α^{(j)}_1 a'_{p_1} + \dots +  α^{(j)}_ℓ a'_{p_ℓ}. 
\end{displaymath}
Alors tous vecteurs
\begin{equation}
  \label{eq:b-20}
  e_j - (α^{(j)}_1 e_{p_1} + \dots +  α^{(j)}_ℓ e_{p_ℓ}) \quad j ∈ \{1,\dots,n \} \setminus  \{p_1, \cdots, p_k \}  
\end{equation}
sont dans le noyaux de $A'$ et alors dans le noyaux de $A$. Cet ensemble de vecteurs est libre. Parce que le nombre total de vecteurs \eqref{eq:b-21} et   \eqref{eq:b-20} est $n$, alors  \eqref{eq:b-21} est une base de l'image de $A'$ et  \eqref{eq:b-20} est une base du noyau de $A'$ (aussi de $A$ respectivement). 

 \begin{example}[Exemple~\ref{example-b2} continué] 
   \label{exe:47}
   On continue avec la matrice
   \begin{displaymath}
     A' =
     \left(\begin{array}{rrrrr}
       1 & 2 & 0 & 1 & 2 \\
       0 & 0 & 2 & 0 & 1 \\
       0 & 0 & 0 & 0 & 0 \\
       0 & 0 & 0 & 0 & 0
     \end{array}\right)
 \end{displaymath}
 Alors $\{ e_1,e_3\}$ est une base de l'image de $A'$ (et de $A$) et
 \begin{displaymath}
   \{ 1 \cdot u_ 1 + 2 \cdot u_ 2 + 2 \cdot u_ 3 + 2 \cdot u_ 4 , 
    0 \cdot u_ 1 + 2 \cdot u_ 2 + 0 \cdot u_ 3 + 2 \cdot u_ 4 \} 
  \end{displaymath}
  est une base de $\im(f)$.
  L'ensemble
  \begin{displaymath}
    \{ e_2 + e_1, e_4 + 2e_1 , e_5 + e_3 + e_1\}  
  \end{displaymath}
  est une base de $\ker(A') = \ker(A)$ et alors
  \begin{displaymath}
     \{ v_2 + v_1, v_4 + 2v_1 , v_5 + v_3 + v_1\}  
   \end{displaymath}
   est une base de noyau $\ker(f)$. 
\end{example}



 
\subsection*{Exercices}


\begin{enumerate}
\item Complétez les omissions  marquées avec $(*)$ pour que l'algorithme de Gauss calcule aussi la matrice de transition $Q ∈ K^{m ×m}$ t.q. $Q ⋅A = A'$ en forme échelonnée. 

  % \begin{algorithm}%[Élimination de Gauss]
  %   \label{alg:4}
  % \caption{Élimination de Gauss avec matrices de transition}
\begin{tabbing}
\textbf{Entrée:} $A ∈ K^{m ×n}$ \\
\textbf{Sortie:}  \= $A'∈ K^{m ×n}$ sous forme échelonnée et $Q ∈ K^{m ×m}$ inversible \\ 
\> vérifiant $Q⋅A = A'$. \\
\pushtabs

$A' := A$ \\
$Q := I_m$ \\
$i := 1$\\
\textbf{tant que} = ($i≤m$) \\
\> trouver le \emph{plus petit} $1 ≤ j ≤n$ tel qu'il existe $k≥i$ avec $a'_{kj} ≠ 0$ \\ 
\> Si un tel élément n'existe pas, alors \textbf{ arrêter} \\
\> échanger les lignes $i$ et $k$ dans $A'$ \\
\> $(*)$ \\
\>  \textbf{pour} \= $k = i+1,\dots, m$ \\
\>            \> soustraire $(a'_{kj}/a'_{ij})$ fois la ligne $i$ de la ligne $k$ dans $A'$ \\
\>            \> $(*)$ \\
\> $i:=i+1$ \\
\poptabs
\textbf{retourner:} $A'$ et $Q$ 
\end{tabbing}
% \end{algorithm}
\item
  \begin{enumerate}[a)] 
  \item  Quel est l'inverse de la matrice $T_{i,j} ∈ K^{n ×n}$, $i,j ∈ \{1,\dots,n\}$, $i ≠j$?
  \item Quel est l'inverse de la matrice $L_{i,j}(α) ∈ K^{n ×n}$, $i,j ∈ \{1,\dots,n\}$, $i ≠j$ et $α ∈K$?
  \item Décrire en mots l'effet de la multiplication de ces matrices à \emph{droite}, c.à.d. étant donnée $A ∈ K^{m × n}$ décrire 
    \begin{displaymath}
A ⋅ T_{i,j} \quad \text{ et  } \quad       A ⋅ L_{i,j}(α). 
    \end{displaymath}
  \end{enumerate}
\item \label{item:33} 
  \begin{enumerate}[a)]
  \item 
  Une matrice $N ∈ K^{n ×n}$ est appelée  \emph{nilpotente} s'il existe un  $k ∈ ℕ_+$ t.q. $N^k = 0$. Montrez que pour une telle matrice nilpotente  
  \begin{displaymath}
    (I +N)^{-1} = I -N + +N^2 - N^3 + \cdots + (-1)^{k-1}N^{k-1}.  \label{item:31}
  \end{displaymath}
\item Basé sur  \ref{item:31}), 
  donner une formule pour l'inverse $R^{-1}$ dune matrice $R ∈ K^{n ×n}$ triangulaire  supérieure avec $1$ sur la diagonale. \label{item:32}
\item Trouver une formule pour l'inverse $R^{-1}$ dune matrice $R ∈ K^{n ×n}$ triangulaire  supérieure avec des éléments diagonales non nuls.

\item Montrer que tout matrice $A ∈ K^{m ×n}$ peut être factorisée comme
  \begin{displaymath}
    U ⋅ A ⋅ V =
    \begin{pmatrix}
      I_k & 0 \\
      0 & 0
    \end{pmatrix}, 
  \end{displaymath}
  où $I_k$ est le rang de $A$ et $U ∈ K^{ m × m}$ et  $V ∈ K^{ n × n}$ sont des matrices inversibles.

  \emph{Indication: Commencer avec la forme échelonnée.  }
\item
   Soient  $V$ et $W$ deux espaces vectoriels sur $ℤ_3$,  $B_1 = \{ v_1,v_2,v_3,v_4\}$ et $B_2 = \{ u_1,u_2,u_3\}$  bases de $V$ et $W$ respectivement. Soit $f : V ⟶ W$ une application linéaire  et les images des $v_i$ comme suivant.
  \begin{eqnarray*}
 f(v_ 1 ) & =&   1 \cdot u_ 1 + 0 \cdot u_ 2 + 2 \cdot u_ 3 + 2 \cdot u_ 4  \\ 
 f(v_ 2 ) & =&   2 \cdot u_ 1 + 1 \cdot u_ 2 + 2 \cdot u_ 3 + 1 \cdot u_ 4  \\ 
 f(v_ 3 ) & =&   0 \cdot u_ 1 + 2 \cdot u_ 2 + 0 \cdot u_ 3 + 2 \cdot u_ 4  \\ 
 f(v_ 4 ) & =&   1 \cdot u_ 1 + 2 \cdot u_ 2 + 1 \cdot u_ 3 + 2 \cdot u_ 4  \\ 
 f(v_ 5 ) & =&   2 \cdot u_ 1 + 2 \cdot u_ 2 + 1 \cdot u_ 3 + 1 \cdot u_ 4 
  \end{eqnarray*}

  Calculer bases du $\ker(f)$ et $\im(f)$. 

\end{enumerate}
\end{enumerate}



%%%Local Variables:
%%% mode: latex
%%% TeX-master: "notes"
%%% ispell-local-dictionary: "french"
%%% End:

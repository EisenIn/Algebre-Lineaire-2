\chapter{Rappel des notions  fondamentales}
\label{cha:noti-fond-et}

\noindent 
On se rappelle de la notion d'un espace vectoriel $V$ sur un corps $K$. Pendant notre cours, on travaille presque exclusivement avec des espaces vectoriels de dimension fini, $\dim(V) = n ∈ ℕ$. Dans ce cas il existe une base finie
\begin{displaymath}
  B = \{ v_1,\dots,v_n\}. 
\end{displaymath}
On considère cette base $B$ comme ordonnée. Alors il y a un premier élément de cet ensemble $B$  un deuxième etc.  


Tout élément $v ∈ V$ possède une représentation unique comme \emph{combinaison linéaire}
\begin{displaymath}
  v = α_1 v_1 + \cdots + α_n v_n, 
\end{displaymath}
où  $α_i ∈ K$ pour $i=1,\dots,n$. Le vecteur
\begin{displaymath}
  [v]_B =
  \begin{pmatrix}
    α_1\\ \vdots \\ α_n
  \end{pmatrix} ∈ Κ^n
\end{displaymath}
est appelé les \emph{coordonnées} de $v$ \emph{relatif à la base} $B$. Si $B'$ est une autre base de $V$ on peut obtenir  les \emph{coordonnées} de $v$ relatif à la base $ℬ'$ à l'aide de la \emph{ matrice de changement de base} $P_{B B'} ∈ K^{n ×n}$. Pour tout $v ∈ V$ on a
\begin{displaymath}
  [v]_{B'} =  P_{BB'} [v]_B. 
\end{displaymath}

\begin{example}
  \label{exe:b-21}
  Supposons  $\dim(V) = 3$ et $B = \{ v_1,v_2,v_3\}$ et $B' = \{ v'_1,v'_2,v'_3\}$ sont deux bases de $V$.  Les $v_i'$ sont représentées comme combinaisons linéaires des $v_i$ comme suivant.
    \begin{eqnarray*}
      v_1' & = &  2 v_1 + v_2 - v_3\\
      v_2' & = &   v_1 + 3v_2 + v_3 \\
      v_3' & = &   v_1 + v_2 -  v_3. 
    \end{eqnarray*}
    La matrice de changement de base $P_{B'B}$ est
    \begin{displaymath}
      P_{B'B} = \left(\begin{array}{rrr}
2 & 1 & 1 \\
1 & 3 & 1 \\
-1 & 1 & -1
\end{array}\right)
\end{displaymath}
On vérifie par exemple 
\begin{displaymath}
  [v_1']_{B'}  = e_1 \text{   et  } [v_1']_{B} = P_{B'B} ⋅ e_1. 
\end{displaymath}
La matrice de changement de base $P_{BB'}$ est
\begin{displaymath}
  P_{BB'} =  P_{B'B}^{-1} = \left(\begin{array}{rrr}
1 & -\frac{1}{2} & \frac{1}{2} \\
0 & \frac{1}{4} & \frac{1}{4} \\
-1 & \frac{3}{4} & -\frac{5}{4}
\end{array}\right).
\end{displaymath}
\end{example}
%
\noindent 
Comme  dans l'exemple~\ref{exe:b-21} le vecteur $e_i ∈ K^n$ est
\begin{displaymath}
  e_i =
  \begin{pmatrix}
    0 \\ \vdots \\ 0 \\ 1 \\ 0 \\ \vdots \\1
  \end{pmatrix}
\end{displaymath}
où la $i$-ième composante est $1$ et les autres sont tous égaux à $0$.  


Soient $V$ et $W$ deux espaces vectoriels sur un corps $K$. Une \emph{application linéaire} est une fonction $f: V ⟶ W$ tel que
\begin{enumerate}[(l.1)] 
\item Pour tous $u,v ∈ V$ on a $f(u+v) = f(u) + f(v)$. 
\item Pour tout $u ∈V$ et $α ∈ K$ on a $f(α u) = α f(u)$. 
\end{enumerate}


\medskip
\noindent 
Si $B_1 = \{v_1,\dots,v_n\}$ est une base de $V$ et  $B_2 = \{u_1,\dots,u_m\}$ est une base de $W$ et si
\begin{displaymath}
  f(v_j) = a_{1j} u_1 + \cdots + a_{mj} u_m 
\end{displaymath}
alors pour tout $v ∈ V$ on vérifie facilement
\begin{displaymath}
  [f(v) ]_{B_2} =
  \begin{pmatrix}
    a_{11} & a_{12} & \cdots & a_{1n} \\
    \vdots &&&\vdots \\
     a_{m1} & a_{m2} & \cdots & a_{mn} \\
  \end{pmatrix}
  [v]_{B_1}.  
\end{displaymath}



\begin{example}
  \label{exe:b2}
  
\end{example}


%%%Local Variables:
%%% mode: latex
%%% TeX-master: "notes"
%%% ispell-local-dictionary: "french"
%%% End:

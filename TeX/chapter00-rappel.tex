\chapter{Rappel des notions  fondamentales}
\label{cha:noti-fond-et}

\noindent 
On se rappelle de la notion d'un espace vectoriel $V$ sur un corps $K$. Pendant notre cours, on travaille presque exclusivement avec des espaces vectoriels de dimension fini, $\dim(V) = n ∈ ℕ$. Dans ce cas il existe une base finie
\begin{displaymath}
  B = \{ v_1,\dots,v_n\}. 
\end{displaymath}
On considère cette base $B$ comme ordonnée. Alors il y a un premier élément de cet ensemble $B$  un deuxième etc.  


Tout élément $v ∈ V$ possède une représentation unique comme \emph{combinaison linéaire}
\begin{displaymath}
  v = α_1 v_1 + \cdots + α_n v_n, 
\end{displaymath}
où  $α_i ∈ K$ pour $i=1,\dots,n$. Le vecteur
\begin{displaymath}
  [v]_B =
  \begin{pmatrix}
    α_1\\ \vdots \\ α_n
  \end{pmatrix} ∈ Κ^n
\end{displaymath}
est appelé les \emph{coordonnées} de $v$ \emph{relatif à la base} $B$. Si $B'$ est une autre base de $V$ on peut obtenir  les \emph{coordonnées} de $v$ relatif à la base $ℬ'$ à l'aide de la \emph{ matrice de changement de base} $P_{B B'} ∈ K^{n ×n}$. Pour tout $v ∈ V$ on a
\begin{displaymath}
  [v]_{B'} =  P_{BB'} [v]_B. 
\end{displaymath}

\begin{example}
  \label{exe:b-21}
  Supposons  $\dim(V) = 3$ et $B = \{ v_1,v_2,v_3\}$ et $B' = \{ v'_1,v'_2,v'_3\}$ sont deux bases de $V$.  Les $v_i'$ sont représentées comme combinaisons linéaires des $v_i$ comme suivant.
    \begin{eqnarray*}
      v_1' & = &  2 v_1 + v_2 - v_3\\
      v_2' & = &   v_1 + 3v_2 + v_3 \\
      v_3' & = &   v_1 + v_2 -  v_3. 
    \end{eqnarray*}
    La matrice de changement de base $P_{B'B}$ est
    \begin{displaymath}
      P_{B'B} = \left(\begin{array}{rrr}
2 & 1 & 1 \\
1 & 3 & 1 \\
-1 & 1 & -1
\end{array}\right)
\end{displaymath}
On vérifie par exemple 
\begin{displaymath}
  [v_1']_{B'}  = e_1 \text{   et  } [v_1']_{B} = P_{B'B} ⋅ e_1. 
\end{displaymath}
La matrice de changement de base $P_{BB'}$ est
\begin{displaymath}
  P_{BB'} =  P_{B'B}^{-1} = \left(\begin{array}{rrr}
1 & -\frac{1}{2} & \frac{1}{2} \\
0 & \frac{1}{4} & \frac{1}{4} \\
-1 & \frac{3}{4} & -\frac{5}{4}
\end{array}\right).
\end{displaymath}
\end{example}
%
\noindent 
Comme  dans l'exemple~\ref{exe:b-21} pour $i ∈ \{1,\dots,n\}$  le vecteur $e_i ∈ K^n$ est
\begin{displaymath}
  e_i =
  \begin{pmatrix}
    0 \\ \vdots \\ 0 \\ 1 \\ 0 \\ \vdots \\1
  \end{pmatrix}
\end{displaymath}
où la $i$-ième composante est $1$ et les autres sont tous égaux à $0$.  


\section{Applications linéaires et matrices}
\label{sec:appl-line-et}
\noindent 
Soient $V$ et $W$ deux espaces vectoriels sur un corps $K$. Une \emph{application linéaire} est une fonction $f: V ⟶ W$ tel que
\begin{enumerate}[(l.1)] 
\item Pour tous $u,v ∈ V$ on a $f(u+v) = f(u) + f(v)$. 
\item Pour tout $u ∈V$ et $α ∈ K$ on a $f(α u) = α f(u)$. 
\end{enumerate}


\medskip
\noindent 
Si $B_1 = \{v_1,\dots,v_n\}$ est une base de $V$ et  $B_2 = \{w_1,\dots,w_m\}$ est une base de $W$ et si
\begin{displaymath}
  f(v_j) = a_{1j} w_1 + \cdots + a_{mj} w_m 
\end{displaymath}
alors pour tout $v ∈ V$ on vérifie facilement
\begin{displaymath}
  [f(v) ]_{B_2} =
  \begin{pmatrix}
    a_{11} & a_{12} & \cdots & a_{1n} \\
    \vdots &&&\vdots \\
     a_{m1} & a_{m2} & \cdots & a_{mn} \\
  \end{pmatrix}
  [v]_{B_1}.  
\end{displaymath}
% La matrice
% \begin{equation}
%   \label{eq:19}
%   A_{B_1,B_2}^f =  \begin{pmatrix}
%     a_{11} & a_{12} & \cdots & a_{1n} \\
%     \vdots &&&\vdots \\
%      a_{m1} & a_{m2} & \cdots & a_{mn} \\
%   \end{pmatrix}
% \end{equation}
% est appelé la matrice de $f$ relatif 



\begin{example}
  \label{example-b2}
  Soient  $V$ et $W$ deux espaces vectoriels sur $ℤ_3$,  $B_1 = \{ v_1,v_2,v_3,v_4\}$ et $B_2 = \{ u_1,u_2,u_3\}$  bases de $V$ et $W$ respectivement. Soit $f : V ⟶ W$ une application linéaire  et les images des $v_i$ comme suivant.
  \begin{eqnarray*}
   f(v_{ 1 }) & =& 1  \cdot w_{ 1 }       +  0  \cdot w_{ 2 }       +  2  \cdot w_{ 3 }       +  1  \cdot w_{ 4 }       \\  
  f(v_{ 2 }) & =& 2  \cdot w_{ 1 }       +  1  \cdot w_{ 2 }       +  0  \cdot w_{ 3 }       +  2  \cdot w_{ 4 }       \\  
  f(v_{ 3 }) & =& 0  \cdot w_{ 1 }       +  2  \cdot w_{ 2 }       +  1  \cdot w_{ 3 }       +  0  \cdot w_{ 4 }       \\  
  f(v_{ 4 }) & =& 1  \cdot w_{ 1 }       +  0  \cdot w_{ 2 }       +  2  \cdot w_{ 3 }       +  1  \cdot w_{ 4 }       \\  
  f(v_{ 5 }) & =& 2  \cdot w_{ 1 }       +  1  \cdot w_{ 2 }       +  0  \cdot w_{ 3 }       +  2  \cdot w_{ 4 }       
  \end{eqnarray*}
  Alors la matrice
  \begin{displaymath}
   A =   \left( \begin{array}{rrrrr} 1 & 2 & 0 & 1 & 2 \\ 0 & 1 & 2 & 0 & 1 \\ 2 & 0 & 1 & 2 & 0 \\ 1 & 2 & 0 & 1 & 2 \end{array}\right) ∈ ℤ_3^{4 × 5} 
 \end{displaymath}
 nous donne pour tout $v ∈ V$ 
 \begin{displaymath}
   [f(v)]_{B_2}  = A [v]_{B_1}. 
 \end{displaymath}
  
\end{example}




\section{L'élimination de Gauss}
\label{sec:analys-gauss-elim}

Rappelons l'élimination de Gauss.

\begin{algorithm}%[Élimination de Gauss]
\label{alg:3}

\begin{tabbing}
Entrée : $A ∈ K^{m ×n}$ \\
Sortie : = $A'∈ K^{m ×n}$ sous forme échelonnée telle qu'il existe une matrice inversible \\
> $Q ∈ K^{m × m}$ vérifiant $Q⋅A = A'$. \\
\pushtabs
\
$A' := A$ \
$i := 1$\
\textbf{tant que} = ($i≤m$) \
> trouver le \emph{plus petit} $1 ≤ j ≤n$ tel qu'il existe $k≥i$ avec $a'{kj} ≠ 0$ \
> Si un tel élément n'existe pas, alors {\bf arrêter} \
> échanger les lignes $i$ et $k$ dans $A'$ \
>  \textbf{pour} = $k = i+1,\dots, m$ \
>            > soustraire $(a'{kj}/a'_{ij})$ fois la ligne $i$ de la ligne $k$ dans $A'$ \
> $i:=i+1$
\poptabs
\end{tabbing}
\end{algorithm}

\noindent
On peut facilement prouver la correction de l'algorithme. Tout d'abord, l'algorithme n'effectue que des opérations élémentaires sur les lignes de la forme
\begin{enumerate}[i)]
\item échanger deux lignes
\item soustraire un multiple d'une ligne à une \textbf{autre} ligne.
\end{enumerate}
Cela signifie que la matrice résultante $A'$ peut être obtenue via
\begin{displaymath}
A' = Q ⋅ A
\end{displaymath}
avec une matrice $Q ∈ ℚ^{m × m}$ non singulière. Par ailleurs, nous avons l'invariant suivant.
\begin{quote}
Après chaque itération de la boucle {\tt tant que}, la matrice $H$ obtenue à partir des $j$ premières colonnes de $A'$ est sous forme échelonnée et les lignes $i, i+1, \dots, m$ de $H$ sont entièrement nulles, voir la figure~\ref{fig:9}.
\end{quote}

\begin{figure}
\centering
\includegraphics[height=5cm]{./Figures/Echelon.jpg}
\caption{Élimination de Gauss : la matrice $A'$ avant la $i$-ème itération de la boucle {\tt tant que}. \label{fig:9}}
\end{figure}

Combien d'opérations arithmétiques l'élimination de Gauss effectue-t-elle ? Soustraire un multiple d'une ligne à une autre ligne dans $A'$ peut se faire en temps $O(n)$. Ainsi, le nombre total d'opérations effectuées dans la boucle {\bf pour} est $O(m ⋅n)$. Il y a $O(m)$ itérations de la boucle {\tt tant que}. Au total, cela montre que l'élimination de Gauss effectue $O(m^2 ⋅n)$ opérations. Mais quelle peut être la taille des nombres au cours de l'algorithme de Gauss ? Est-il possible que des nombres dont la longueur de codage binaire n'est pas polynomiale par rapport à la longueur de codage totale de la matrice $A$ doivent être manipulés ? Heureusement, la réponse à cette question est « Non ». Pour apporter cette réponse, nous devons démontrer ce qui suit.

%%%Local Variables:
%%% mode: latex
%%% TeX-master: "notes"
%%% ispell-local-dictionary: "french"
%%% End:

%----------------------------------------------------------------------------------------
%	PACKAGES AND THEMES
%----------------------------------------------------------------------------------------
\documentclass[aspectratio=169,xcolor=dvipsnames,noframenumbering]{beamer}

\usetheme{SimplePlusAIC}

\newcommand\myemph[1]{{\color{darkred} #1}}
\newcommand\platz{\vspace{.7cm}}
\newcommand\tabb{\hspace{.75cm}}

\usepackage{mathrsfs}
%\usepackage{ulem}

\usepackage{color}
\definecolor{navy}{RGB}{0,0,205}
\definecolor{NavyBlue}{RGB}{0,0,205}
\definecolor{darkred}{RGB}{178,34,34}
\definecolor{darkblue}{RGB}{0,10,230}
\definecolor{green}{RGB}{20,180,20}
\definecolor{titleblue}{rgb}{0,0,0.3}
\definecolor{mediumblue}{rgb}{0.5,0.5,1}
\definecolor{lightblue}{cmyk}{0.2,0.1,0,0}
\definecolor{darkgreen}{rgb}{0.0,0.5,0}

\newcommand{\mnred}[1]{{\color{red}#1}}
\newcommand{\mred}[1]{{\color{darkred}#1}}
\newcommand{\mgreen}[1]{{\color{green}#1}}

\newcommand{\scriptgray}[1]{{ \scriptsize \color{gray}#1}}

\DeclareMathOperator{\lcm}{lcm}

\newcommand{\toright}[1]{{ \hfill{\scriptgray{#1}}}}

\newcommand{\cone}{\operatorname{cone}}
\newcommand{\intcone}{\operatorname{intcone}}
\newcommand{\poly}{\operatorname{poly}}
\newcommand{\rank}{\operatorname{Rank}}

\renewcommand\geq{\geqslant}
\usepackage{hyperref}
\usepackage{graphicx} % Allows including images
\usepackage{booktabs} % Allows the use of \toprule, \midrule and  \bottomrule in tables  
\usepackage{svg} %allows using svg figures
\usepackage{tikz}
\usetikzlibrary{arrows.meta,patterns}
\usetikzlibrary{ipe} % ipe compatibility library

\tikzstyle{ipe stylesheet} = [
  ipe import,
  even odd rule,
  line join=round,
  line cap=butt,
  ipe pen normal/.style={line width=0.4},
  ipe pen heavier/.style={line width=0.8},
  ipe pen fat/.style={line width=1.2},
  ipe pen ultrafat/.style={line width=2},
  ipe pen normal,
  ipe mark normal/.style={ipe mark scale=3},
  ipe mark large/.style={ipe mark scale=5},
  ipe mark small/.style={ipe mark scale=2},
  ipe mark tiny/.style={ipe mark scale=1.1},
  ipe mark normal,
  /pgf/arrow keys/.cd,
  ipe arrow normal/.style={scale=7},
  ipe arrow large/.style={scale=10},
  ipe arrow small/.style={scale=5},
  ipe arrow tiny/.style={scale=3},
  ipe arrow normal,
  /tikz/.cd,
  ipe arrows, % update arrows
  <->/.tip = ipe normal,
  ipe dash normal/.style={dash pattern=},
  ipe dash dotted/.style={dash pattern=on 1bp off 3bp},
  ipe dash dashed/.style={dash pattern=on 4bp off 4bp},
  ipe dash dash dotted/.style={dash pattern=on 4bp off 2bp on 1bp off 2bp},
  ipe dash dash dot dotted/.style={dash pattern=on 4bp off 2bp on 1bp off 2bp on 1bp off 2bp},
  ipe dash normal,
  ipe node/.append style={font=\normalsize},
  ipe stretch normal/.style={ipe node stretch=1},
  ipe stretch normal,
  ipe opacity 10/.style={opacity=0.1},
  ipe opacity 30/.style={opacity=0.3},
  ipe opacity 50/.style={opacity=0.5},
  ipe opacity 75/.style={opacity=0.75},
  ipe opacity opaque/.style={opacity=1},
  ipe opacity opaque,
]

\definecolor{red}{rgb}{1,0,0}
\definecolor{blue}{rgb}{0,0,1}
\definecolor{green}{rgb}{0,1,0}
\definecolor{yellow}{rgb}{1,1,0}
\definecolor{orange}{rgb}{1,0.647,0}
\definecolor{gold}{rgb}{1,0.843,0}
\definecolor{purple}{rgb}{0.627,0.125,0.941}
\definecolor{gray}{rgb}{0.745,0.745,0.745}
\definecolor{brown}{rgb}{0.647,0.165,0.165}
\definecolor{navy}{rgb}{0,0,0.502}
\definecolor{pink}{rgb}{1,0.753,0.796}
\definecolor{seagreen}{rgb}{0.18,0.545,0.341}
\definecolor{turquoise}{rgb}{0.251,0.878,0.816}
\definecolor{violet}{rgb}{0.933,0.51,0.933}
\definecolor{darkblue}{rgb}{0,0,0.545}
\definecolor{darkcyan}{rgb}{0,0.545,0.545}
\definecolor{darkgray}{rgb}{0.663,0.663,0.663}
\definecolor{darkgreen}{rgb}{0,0.392,0}
\definecolor{darkmagenta}{rgb}{0.545,0,0.545}
\definecolor{darkorange}{rgb}{1,0.549,0}
\definecolor{darkred}{rgb}{0.545,0,0}
\definecolor{lightblue}{rgb}{0.678,0.847,0.902}
\definecolor{lightcyan}{rgb}{0.878,1,1}
\definecolor{lightgray}{rgb}{0.827,0.827,0.827}
\definecolor{lightgreen}{rgb}{0.565,0.933,0.565}
\definecolor{lightyellow}{rgb}{1,1,0.878}
\definecolor{black}{rgb}{0,0,0}
\definecolor{white}{rgb}{1,1,1} 

\usepackage[utf8]{inputenc} 
\usepackage{utf8math}


\usepackage[round]{natbib}   % omit 'round' option if you prefer square brackets
\bibliographystyle{plainnat}


\usepackage{verbatim}
\usetikzlibrary{arrows,shapes} 
\usepackage{makecell}
\usepackage{amsmath}
\usepackage{mathtools}
\usepackage{caption}
\usepackage{subcaption}
\newcommand*{\defeq}{\stackrel{\text{def}}{=}}
\newcommand{\N}{\mathbb{N}}
\newcommand{\Z}{\mathbb{Z}}

\newcommand{\E}{\mathbb{E}}
\newcommand{\B}{\mathcal{B}}
\newcommand{\C}{\mathcal{C}}
\newcommand{\D}{\mathcal{D}}
\newcommand{\Polyhedre}{\mathcal{P}}
\newcommand{\Set}{\mathcal{S}}
\newcommand{\Lcone}{\mathbb{L}}
\newcommand{\Lapprox}{\mathcal{L}}
\newcommand{\epl}{\varepsilon}
\DeclarePairedDelimiter\ceil{\lceil}{\rceil}
\DeclarePairedDelimiter\floor{\lfloor}{\rfloor}
\makeatletter
\DeclareRobustCommand*{\bfseries}{%
  \not@math@alphabet\bfseries\mathbf
  \fontseries\bfdefault\selectfont
  \boldmath
}
\makeatother 
\makeatletter
\newcommand{\Pause}[1][]{\unless\ifmeasuring@\relax
\pause[#1]%
\fi}
\makeatother
\tikzset{hide on/.code={\only<#1>{\color{white}}}}
\tikzstyle{flowchart} = [rectangle, rounded corners, minimum width=3cm, minimum height=1cm,text centered, draw=black]
%Select the Epilogue font (requires luaLatex or XeLaTex compilers)

\usepackage{tgadventor}
\renewcommand*\familydefault{\sfdefault} %% Only if the base font of the document is to be sans serif
\usepackage[T1]{fontenc}





    
%----------------------------------------------------------------------------------------
%	TITLE PAGE
%----------------------------------------------------------------------------------------

\title{Algèbre linéaire avancée II - diagonalisation}
\subtitle{ MATH-115(a) }

\author[Eisenbrand]{Fritz Eisenbrand}

 
 

\date{}
%----------------------------------------------------------------------------------------
%	PRESENTATION SLIDES
%----------------------------------------------------------------------------------------

\begin{document}

\begin{frame}[plain]
    % Print the title page as the first slide
    \titlepage
\end{frame}

\pagestyle{empty} 

 \begin{frame}{Informations sur} 

  \begin{alertblock}{}

    \begin{columns}
      \begin{column}{.5\textwidth}
        \begin{itemize}
        \item Contenue du cours
         \item Exercices et examen 
         \item Contact et forum
         \end{itemize}
       \end{column}
       \begin{column}{.5\textwidth}
         %  \begin{itemize}
         % \item Contenue du cours
         % \item Exercices et examen 
         % \item Contact et forum
         % \end{itemize}
       \end{column}
       
    \end{columns}
      
     \end{alertblock}
          \begin{center}
       \input{./Images/projection-on-plane.tex}       
     \end{center}    
    
  
   \end{frame}
   



   \begin{frame}

     \begin{columns}
       \begin{column}{0.4\textwidth }                 
     \frametitle{Assistants}
     Assistant principal:
     \begin{itemize}
     \item Claudio Pfammatter 
     \end{itemize}

     \bigskip 
     
     Assistants doctorants:
     \begin{itemize}

     \item Eleonore Bach \item Mariana Martinez \item  Niklas Schmitz \item  Emre Özavci \item Michele Barruca 
     \end{itemize}
   \end{column}
   \begin{column}{0.6\textwidth }         
     Assistants étudiants      
     \begin{itemize}\item 
       Aleksandra Bogdanova 
     \item Benoit Cuenot 
     \item Emna Boughizane
     \item Leonard Lebrun
     \item  Patrick-Cristian Dan
     \item Rafael Sierra 
     \item Sandro Pfammatter
     \item Solveig Girardin
     \end{itemize}
   \end{column}
   
\end{columns}
\end{frame}

   \begin{frame}{Contenue}
   
     \begin{enumerate}
     \item Polynômes: {\small Racines, division avec reste,
         reconstruction, algorithme d'Euclide, Factorisation en
         irréductibles }
     \item Algorithmes et leur analyse:  {\small Déterminant, équations linéaires, polynôme caractéristique, diagonalisation }
     \item  Formes bilinéaires: {\small théorème de Sylvester, produits scalaires, norme, orthogonalisation, moindre carrées }
     \item Théorème spectral: {\small  valeurs singulières, pseudo-inverses}
     \item Systèmes différentiels linéaires: {\small exponentiel d'une matrice}
     \item La forme normale de Jordan
     \item  Algèbre linéaire sur les entiers{ \small structure de groupes abélien  engendrés finis }
   \end{enumerate}

   \end{frame}

   \begin{frame}{Notes du cours}

     \bigskip 
     
     Se trouvent sur moodle

     \bigskip 
     {Régulièrement}  mis à jours.

     \bigskip 
     {Fin du semestre}: Reflètent le contenu  relevant  
   
     
   \end{frame}
   


   \begin{frame}
     \frametitle{Exercices}

     \begin{itemize}
     \item Une série chaque semaine sur moodle.
     \item Première à télécharger  aujourd'hui 
     \item Exercice $+$ et $*$: \myemph{Pas} de corrigé. Seront \myemph{discutés} lors des sessions du mardi.
     \item Série de la semaine $n$
       \begin{itemize}
       \item Ex. $+$: Exercice préparatoire discuté Mardi semaine $n$
       \item Ex. $*$: Exercice type question ouverte examen   discuté Mardi semaine $n+1$
       \item Aujourd'hui: On discute exercice $+$ semaine $1$
       \item Autres exercices: Travaille pendant séance vendredi, semaine $n$.
       \item Ce vendredi: On discute exercices semaine $1$
       \end{itemize}
     \item Séance vendredi:
       \begin{itemize}
       \item Répartition des salles affichée sur moodle (selon nom)
       \item 14 assistants  à disposition pour questions, aide, discussions. 
       \end{itemize}
       
     \end{itemize}
     
   \end{frame}

   \begin{frame}
     \frametitle{Examen}

     \begin{itemize}
     \item 3-4 questions \myemph{ouvertes} (type ex. $*$). Une des trois est exercice $*$ original des séries
     \item $∼$ 7 questions  \myemph{choix multiples} 
     \item  $∼$ 14 questions \myemph{vrai/faux}     
     \end{itemize}

     \bigskip

     Examens des années précédentes  \myemph{sur moodle}
   \end{frame}

    \begin{frame}
     \frametitle{Take home exam} 
     \begin{itemize}
     \item 3 examens \myemph{take home} pendent le semestre
     \item Chaque représente $∼$ 1/3 d'un examen (une question ouverte, quelque CM + V/F)
     \item V/F et CM sont corrigés automatiquement
     \item Question ouverte à rendre pour feedback (en latex)  \myemph{fortement conseillé} 
     \item Dates:
       \begin{itemize}
       \item 10.03 - 16.03
       \item 07.04 - 13.04
       \item 12.05 - 18.05
       \end{itemize}
     \item Ne \myemph{comptent pas} vers la note finale 
     \end{itemize}
   \end{frame}


   
   \begin{frame}{Contact et communication}

     \begin{itemize}
     \item Questions, commentaires: \myemph{Utilisez forum!}
     \item Assistants sont présent sur forum 
     \item \myemph{Formatez les maths en latex!} 
     \item Ne m'envoyez pas de  e-mail!
     \item Essayez de résoudre les exercices soigneusement. Donnez vous assez de temps. 
     \item Office hours: Mardi 16:00 - 17:00.

       
      R.d.v. impératif: e-mail  {\tt pauline.bataillard@epfl.ch  }
     \end{itemize}
     
   \end{frame}


   

%%% Local Variables:
%%% mode: latex
%%% TeX-master: "Slides"
%%% End:

 
\end{document}


%%%Local Variables:
%%% mode: latex
%%% TeX-master: t
%%% TeX-engine: xetex
%%% End:

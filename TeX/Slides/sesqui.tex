\pagestyle{empty} 

\begin{frame}{}

  Soit $V$
  un espace vectoriel sur  $\C$ et 
  \begin{displaymath}
    \pscal{,} : V × V ⟶ ℂ 
  \end{displaymath}
  
  \begin{enumerate}[PH 1]
  \item On a $\pscal{v,w} = \overline{\pscal{w,v}}$ pour tout $v,w \in V$.  \label{ph1}
  \item Si $u,v$ et $w$ sont des éléments de $V$,  \label{ph2}
    \begin{displaymath}
      \pscal{u,v+w} = \pscal{u,v}+\pscal{u,w} \, \text{ et } \,  \pscal{v+w,u} = \pscal{v,u}+\pscal{w,u}
    \end{displaymath}
  \item Si $x \in \C$ et $u,v \in V$,  \label{ph3}
    \begin{displaymath}
       \pscal{x \cdot u , v} = x \pscal{u,v} \, \text{ et } \,   \pscal{u , x \cdot v} = \overline{x}\cdot  \pscal{u,v}.
    \end{displaymath}  
  \end{enumerate}
  \noindent
on dit que $\pscal{,}$ est 
\begin{enumerate}[i)]
\item une \emph{forme sesquilinéaire}, si $\pscal{,}$ satisfait 
  PH~\ref{ph2} et PH~\ref{ph3}. 
\item une \emph{forme hermitienne}, si $\pscal{,}$ satisfait 
  PH~\ref{ph1}, PH~\ref{ph2} et PH~\ref{ph3}. 
\item un \emph{produit hermitien}, si   $\pscal{,}$ satisfait 
  PH~\ref{ph1}, PH~\ref{ph2} et PH~\ref{ph3} et 
  \begin{displaymath}
    \pscal{v,v} >0, \text{ pour tout } v \in V \setminus \{0\}. 
  \end{displaymath}
\end{enumerate}

    
  
\end{frame}


\begin{frame}
  
Une forme  sesquilinéaire  est \emph{non dégénérée à gauche} si la condition suivante est vérifiée. 
\begin{quote}
  Si $v \in V$ et si $\pscal{v,w}=0$ pour tout $w \in V$, alors $v = 0$. 
\end{quote}

\end{frame}
   




\begin{frame}
  \begin{definition}
    Si $\pscal{,} $  est une forme hermitienne, 
pour tout $v \in V$, on a $\pscal{v,v} \in \R$ dès que $\pscal{v,v} = \overline{\pscal{v,v}}$ par PH~\ref{ph1}.  On dit que la forme hermitienne   est \emph{définie positif} si $\pscal{v,v} >0$ pour tout $v \in V \setminus\{0\}$. Alors un produit hermitien est une forme hermitienne définie positif. 
  \end{definition}
\end{frame}
 


\begin{frame}

  
\begin{example}
  \label{exe:12}
  Le produit \emph{hermitien standard} de $\C^n$ 
  \begin{displaymath}
    \pscal{u,v} = \sum_i u_i \overline{v_i}
  \end{displaymath}
  satisfait les condition PH~\ref{ph1}-\ref{ph3} et est défini positif.  
\end{example}



\end{frame}



\begin{frame}
  
Les notions d'\emph{orthogonalité, de perpendicularité, de base orthogonale} et \emph{de supplémentaire orthogonal}  sont définies comme avant. Aussi les  \emph{coefficients de Fourier} sont les mêmes.
Soit $w ∈ V$, $w ≠0$ et $v ∈V$. On cherche $α ∈ ℂ$ satisfaisant
\begin{displaymath}
  〈 w, v - α w 〉 = 0. 
\end{displaymath}
On trouve $\wb{α} = 〈w,v〉 / 〈 w,w〉$. Puisque $〈 w,w〉 ∈ ℝ$ et  $\wb{〈w,v〉} =  〈v,w〉$, alors
\begin{displaymath}
  α = 〈v,w〉 / 〈 w,w〉. 
\end{displaymath}
\end{frame}

\begin{frame}
  
Soient $V$ un espace vectoriel sur $\C$ de dimension finie et $f: V \times V \longrightarrow \C$ une forme sesquilinéaire. Pour une base  $B = \{v_1,\dots,v_n\}$ de $V$ et $x = \sum_i \alpha_i v_i$ et $y = \sum_i \beta_i v_i$ on a 
\begin{displaymath}
  \pscal{x,y} = \sum_{ij} \alpha_i\overline{\beta_j} f(v_i,v_j)
\end{displaymath}
\end{frame}


\begin{frame}
  
\end{frame}




\begin{frame}

  \begin{definition}
  \label{def:17}
  Une matrice $A \in \C^{n \times n}$ est appelée \emph{hermitienne} si on a 
  \begin{displaymath}
    A = \overline{A^T}. 
  \end{displaymath}
\end{definition}

\end{frame}


\begin{frame}


\begin{lemma}
  \label{prop:3}
  Soit  $V$  un espace vectoriel sur $\C$ de dimension finie et soit $B$  une base de $V$. Une forme sesquilinéaire $f$ est une forme hermitienne si et seulement si $A_B^f$ est hermitienne.  
\end{lemma}

  
\end{frame}


\begin{frame}

  
\end{frame}



\begin{frame}

  
\begin{definition}
  \label{def:18}
  Deux matrices $A,B \in \C^{n \times n}$ sont \emph{congruentes complexes} s'il existe une matrice inversible $P \in \C^{n \times n}$ telle que $A = {P^T} \cdot B \cdot \overline{P}$. Nous écrivons $A \cong_\C B$.  
\end{definition}
\end{frame}



\begin{frame}
  
\begin{theorem}
  \label{thr:11}
  Soit $V$ un espace vectoriel sur $\C$ de dimension finie, muni d'une forme  hermitienne. Alors $V$ possède une base orthogonale. 
\end{theorem}
\end{frame}



\begin{frame}

  
\end{frame}


\begin{frame}{Exemple}
  \scriptsize

  


  On considère la matrice hermitienne 
  \begin{displaymath}
    A = \left[\begin{matrix}0 & - i & 3 + 4 i\\i & -2 & 12\\3 - 4 i & 12 & 5\end{matrix}\right]
  \end{displaymath}
et le but est de trouver une matrice inversible $P \in \C^{3 \times 3}$ telle que 
\begin{displaymath}
  P^T \cdot A \cdot \overline{P}
\end{displaymath}
est une matrice diagonale. Nous échangeons la première et la deuxième colonne ainsi que la première et la deuxième ligne et obtenons 
\begin{displaymath}
\left[\begin{matrix}-2 & i & 12\\- i & 0 & 3 + 4 i\\12 & 3 - 4 i & 5\end{matrix}\right]. 
\end{displaymath}
Après on transforme 
\begin{displaymath}
\left[\begin{matrix}1 & 0 & 0\\- 0.5 i & 1 & 0\\6 & 0 & 1\end{matrix}\right]\cdot  \left[\begin{matrix}-2 & i & 12\\- i & 0 & 3 + 4 i\\12 & 3 - 4 i & 5\end{matrix}\right]\cdot  
\left[\begin{matrix}1 & 0.5 i & 6\\0 & 1 & 0\\0 & 0 & 1\end{matrix}\right]
  = 
\left[\begin{matrix}-2 & 0 & 0\\0 & 0.5 & 3 - 2 i\\0 & 3 + 2 i & 77\end{matrix}\right]
\end{displaymath}

\end{frame}
\begin{frame}
  \scriptsize 
La prochaine transformation est 

\begin{displaymath}
  \left[\begin{matrix}1 & 0 & 0\\0 & 1 & 0\\0 & -6 - 4 i & 1\end{matrix}\right] \cdot 
\left[\begin{matrix}-2 & 0 & 0\\0 & 0.5 & 3 - 2 i\\0 & 3 + 2 i & 77\end{matrix}\right] \cdot 
\left[\begin{matrix}1 & 0 & 0\\0 & 1 & -6 + 4 i\\0 & 0 & 1\end{matrix}\right] = 
\left[\begin{matrix}-2 & 0 & 0\\0 & 0.5 & 0\\0 & 0 & 51\end{matrix}\right]. 
\end{displaymath}
Pour 
\begin{displaymath}
P =   \left[\begin{matrix}0 & 1 & 0\\1 & 0 & 0\\0 & 0 & 1\end{matrix}\right] \cdot 
\left[\begin{matrix}1 & - 0.5 i & 6\\0 & 1 & 0\\0 & 0 & 1\end{matrix}\right]
\cdot 
\left[\begin{matrix}1 & 0 & 0\\0 & 1 & -6 - 4 i\\0 & 0 & 1\end{matrix}\right]
\end{displaymath}
on obtient 
\begin{displaymath}
  P^T \cdot A \cdot \overline{P} = \left[\begin{matrix}-2 & 0 & 0\\0 & 0.5 & 0\\0 & 0 & 51\end{matrix}\right]. 
\end{displaymath}



  
\end{frame}


\begin{frame}{Espaces hermitiens}


\begin{definition}
  \label{def:h5}
  Soit $\pscal{,}$ un produit hermitien. La \emph{longueur} ou la \emph{norme} d'un élément $v \in V$ est le nombre 
  \begin{displaymath}
    \| v \| = \sqrt{\pscal{v,v}}.
  \end{displaymath}
  Un élément $v \in V$ est un \emph{vecteur unitaire} si $\|v\| = 1$. 
\end{definition}

  
\end{frame}


\begin{frame}{Les propriétés suivantes sont facilement vérifiées}
\begin{enumerate}[i)]
\item Pour tout $v \in V$, $\|v\|\geq 0$ et $\|v\| = 0$ si et seulement si $v = 0$. \label{item:8}
\item Pour $\alpha \in \C$ et $v \in V$ on a $\| \alpha \cdot v \| = |\alpha| \cdot \|v\|$. \label{item:9}
\item Pour chaque $u,v \in V$ $\|u+v\| \leq \|u\| + \|v\|$. \label{tr:2}
\end{enumerate}

\end{frame}



\begin{frame}{Procédé de Gram-Schmidt }

\begin{theorem}
  Soit $V$ un espace hermitien et  $\{v_1,\dots,v_n\} \subseteq V$
  un ensemble libre.  
  Il existe un ensemble libre orthogonal $\{u_1,\dots,u_n\}$
  de $V$
  tel que pour tout $i$,
  $\{v_1,\dots,v_i\}$
  et $\{u_1,\dots,u_i\}$ engendrent le même sous-espace de $V$.
\end{theorem}
\end{frame}



\begin{frame}
  
\end{frame}


\begin{frame}{}

\begin{corollary}
  \label{co:6}
  Soit $V$
  un espace hermitien de dimension finie.  $V$
  possède alors une base orthonormale.
\end{corollary}

\end{frame}
%%% Local Variables:
%%% mode: latex
%%% TeX-master: "Slides"
%%% End:

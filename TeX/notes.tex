\documentclass[a4paper,11pt,french]{scrbook} 
\usepackage[utf8]{inputenc} 
\usepackage{utf8math}
\usepackage[boldsans]{concmath}
\usepackage{tabularx}
\usepackage{mathrsfs}
\usepackage{nicematrix} %
\usepackage[french]{babel}
\frenchbsetup{IndentFirst=false}
\usepackage{mathtools} % includes amsmath
\usepackage{amssymb}
\usepackage{amsthm}
\usepackage{amscd}
\usepackage{todonotes}
\usepackage{hyperref}
\usepackage{enumerate}

%\usepackage{tikz}
\usepackage{framed}
%\usepackage[colorlinks]{hyperref}

%\usepackage{showlabels}  
\usepackage[Algorithme]{algorithm}
\usepackage{algorithmicx,algpseudocode} 
 

%%%%%%%%%%% Python Listing 
\usepackage{listings}
\usepackage{xcolor} 
\definecolor{codegreen}{rgb}{0,0.6,0}
\definecolor{codegray}{rgb}{0.5,0.5,0.5}
\definecolor{codepurple}{rgb}{0.58,0,0.82}
\definecolor{backcolour}{rgb}{0.95,0.95,0.92}
 
\lstdefinestyle{mystyle}{
    backgroundcolor=\color{backcolour},   
    commentstyle=\color{codegreen},
    keywordstyle=\color{magenta},
    numberstyle=\tiny\color{codegray},
    stringstyle=\color{codepurple},
    basicstyle=\footnotesize,
    breakatwhitespace=false,         
    breaklines=true,                 
    captionpos=b,                    
    keepspaces=true,                 
    numbers=left,                    
    numbersep=5pt,                  
    showspaces=false,                
    showstringspaces=false,
    showtabs=false,                  
    tabsize=2
  } 

\lstset{style=mystyle}  
%%%%%%%%%%%%%%%%%%%%%%%


% %%%%%%%%%%%%% Margin Notes

 \usepackage{marginnote}
% \usepackage[top=1.5cm, bottom=1.5cm, outer=5cm, inner=2cm, heightrounded, marginparwidth=2.5cm, marginparsep=2cm]{geometry}
% %%%%%%%%%%%%%%%%%%%%%%%%%%%%

\newcommand{\smat}[1]{ \big(\begin{smallmatrix} #1 \end{smallmatrix}\big)}

\newcommand{\E}{\mathbb{E}}
\newcommand{\N}{\mathbb{N}}
\newcommand{\Q}{\mathbb{Q}}
\newcommand{\R}{\mathbb{R}}
\newcommand{\Z}{\mathbb{Z}}
\newcommand{\C}{\mathbb{C}}
\newcommand{\K}{\mathbb{K}}
\newcommand{\x}{\mathbf{x}}
\newcommand{\y}{\mathbf{y}}
\newcommand{\X}{\mathscr{X}}
\newcommand{\cA}{\mathcal{A}}
\newcommand{\cD}{\mathcal{D}}
\newcommand{\cI}{\mathcal{I}}
\newcommand{\cP}{\mathcal{P}}
\newcommand{\cV}{\mathcal{V}}
\newcommand{\pscal}[1]{\langle {#1} \rangle}
\newcommand{\wt}[1]{\widetilde{#1}}
\newcommand{\wb}[1]{\overline{#1}}
\newcommand{\car}{\mathrm{Char}}

\providecommand{\one}{\mathbf{1}}
\DeclareMathOperator{\id}{id}
\DeclareMathOperator{\vol}{vol}
\DeclareMathOperator{\rank}{rang}
\DeclareMathOperator{\ligne}{ligne}
\DeclareMathOperator{\im}{im}
\DeclareMathOperator{\noy}{noyau}
\DeclareMathOperator{\cone}{cone}
\DeclareMathOperator{\tcone}{tcone}
\DeclareMathOperator{\conv}{conv}
\DeclareMathOperator{\spec}{spec}
\DeclareMathOperator{\cof}{cof}
\DeclareMathOperator{\diam}{diam}
\DeclareMathOperator{\sign}{sgn}
\DeclareMathOperator{\poly}{poly}
\DeclareMathOperator{\Ker}{Ker}
\DeclareMathOperator{\Var}{Var}
\DeclareMathOperator{\Id}{Id}
\DeclareMathOperator{\tr}{tr}
\DeclareMathOperator{\relint}{relint}
\DeclareMathOperator{\spa}{span}
\DeclareMathOperator{\Tr}{Tr}
\DeclareMathOperator{\diag}{diag}
\DeclarePairedDelimiter\mnorm{\lvert\lvert\lvert}{\rvert\rvert\rvert} % matrix norm


\theoremstyle{plain}
\newtheorem{theorem}{Théorème}[chapter]
\newtheorem{lemma}[theorem]{Lemme}
\newtheorem{proposition}[theorem]{Proposition}
\newtheorem{claim}[theorem]{Claim}
\newtheorem{corollary}[theorem]{Corollaire}

\theoremstyle{definition}
\newtheorem{definition}{Définition}[chapter]
\newtheorem*{notation}{Notation}
\newtheorem{example}{Exemple}[chapter]
\newtheorem{remark}[theorem]{Remarque}
\newtheorem{problem}{Problème}[chapter]
\newtheorem{exercise}{Exercice}[chapter]
%\newtheorem{algorithm}{Algorithme}[chapter]



\newcommand{\iunit}{\mathrm{i}}
\newcommand{\CC}{{\mathbb C}}
\newcommand{\EE}{{\mathbb E}}
\newcommand{\FF}{{\mathbb F}}
\newcommand{\KK}{{\mathbb K}}
\newcommand{\NN}{{\mathbb N}}
\newcommand{\QQ}{{\mathbb Q}}
\newcommand{\RR}{{\mathbb R}}
\newcommand{\ZZ}{{\mathbb Z}}

\newcommand{\calB}{{\mathcal B}}



\title{Algèbre linéaire avancée II}
\author{Friedrich Eisenbrand}

%\includeonly{chapter00-rappel.tex}

\begin{document}

\maketitle
  
\section*{Préface}
\noindent Ceci sont mes notes du cours \emph{Algèbre Linéaire Avancée II}.
La qualité de ce texte dépend fortement de la participation  des étudiants. 
Ces  sources sont gérées sur \emph{GitHub}, une plateforme importante de collaboration. Si vous trouvez des fautes, des erreurs typographiques, ou même des démonstrations plus élégantes, ou des exemples qui vous aident à comprendre la matière, je vous invite à créer une \emph{branch} des fichiers en question, où dedans vous éditez le texte. Après, vous publiez (\emph{publish}) cette \emph{branch} et vos collègues peuvent discuter vos modifications. Si vous êtes satisfait avec vos modifications, vous me demandez, avec un \emph{Pull Request}, d'accepter vos modifications et finalement, le document peut être changé. Je me réjouis en avance de votre participation. 

  
\section*{Contributions}

Des corrections et modifications ont été implémentées par: 
\begin{itemize}
\item Orane Jecker 
\item Natalia Karaskova
\item Dylan Samuelian
\item Aziz Benmosbah
\item Djian Post
\item Robin Mamie
\item Alfonso Cevallos
\item Kévin Jorand
\item Charles Dufour 
\item Christoph Hunkenschröder
\item Adam Cierniak
\item Mann-Tchi Dang
\item Yasmine Bennis
\item Corentin 
\item Lucas Gehrt
\item Jooyoung Kim 
\item Léo Navarro
\item Arthur Serres
\item Matthieu Haeberle
\item Chady Bensaid
\item Belarbi Zied
\item Sébastien de Morsier 
\item Cedric Lehr
\item Orso Renucci
\item Anselm Albrecher
\item Beatrice Serrurier
\item Pablo Habib
\item Vittorio Sandri
\item Youssef Jamali
\end{itemize}

Le deuxième   chapitre sur les valeurs propres est basé, en partie, sur les notes du cours de Daniel Kressner.  

\tableofcontents  
\chapter{Notions fondamentales et notation}
\label{cha:noti-fond-et}




%%%Local Variables:
%%% mode: latex
%%% TeX-master: "notes"
%%% ispell-local-dictionary: "french"
%%% End:

\include{chapter01-polynomes}
\include{chapter02-valeurs-propres} 
\include{chapter03-produits-scalaires}
\include{chapter04-theoreme-spectral}
\include{chapter05-sys-diff-lin}
\include{chapter06-entiers}
\include{chapter07-groupes}

\bibliographystyle{alpha}
\bibliography{books}
\end{document}


%%% Local Variables:
%%% mode: latex
%%% TeX-master: t
%%% ispell-local-dictionary: "french"
%%% End:


